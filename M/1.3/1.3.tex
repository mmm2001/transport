\documentclass[../../main/main.tex]{subfiles} 
\begin{document}

\section {Гиперболические системы уравнений в частных производных первого порядка. Инварианты
Римана. Корректная постановка краевых условий для гиперболических систем уравнений}
% 
Рассмотрим систему одномерных дифференциальных уравнений первого порядка вида:

\begin{equation*}
    \frac{\partial \vec{u}}{\partial t}+A \frac{\partial \vec{u}}{\partial x}=\vec{f}
\end{equation*}
где $\vec{u}=(u_1, u_2, u_3)$, матрица $A$ - с постоянными коэффициентами  

\textbf{\underline{Определение}} Система является гиперболической в том случае, если матрица A обладает полным набором
 левых собственных векторов  (различные вещественные собственные числа различны(строгая гиперболичность))

 \begin{equation*}
    \exists \quad  \{\lambda_i, \vec{e}_i\}_{i=1}^n \quad \vec{e}^T_iA=\lambda_i\vec{e}^T
 \end{equation*}

 Тогда исходная система приводится к несвязных линейных уравнений переноса относительно новых переменных - \textbf{инваринатов Римана}

 \begin{equation*}
    \textbf{E}\frac{\partial \vec{u}}{\partial t}+\textbf{E}A \frac{\partial \vec{u}}{\partial x}=\textbf{E}\vec{f}
 \end{equation*}

 \begin{equation*}
    \frac{\partial \textbf{E} \vec{u}}{\partial t}+\Lambda\frac{\partial \textbf{E}\vec{u}}{\partial x}=\textbf{E}\vec{f}  
 \end{equation*}

где $\textbf{E}$ - матрица строки которой составленны из собственных векторов матрицы $\textbf{A}$, $\Lambda$ - диагональная матрица с элементами 
на диагонали - собственные занчения $\textbf{A}$\\


Переменные задаваемые соотоноешением $R=(R_1, R_2, R_3) = \mathbf{E}\vec{u}$ - инварианты Римана

\subsection{Граничные условия}
Задача является поставленно корректно, если заданы начальные условия для всех компонент вектора $\vec{u}$ и граничные
условия удволеьворяющие:
\begin{itemize}
    \item На левой границе  - условия и значения инвариантов приходящих с правой границы должны давать возможность 
     определить значения инвариантов уходящих с левой границы  внутрь расчетной области
     \item На правой границе  - условия и значения инвариантов приходящих с левой границы должны давать возможность 
     определить значения инвариантов уходящих с правой границы внутрь расчетной области
\end{itemize}

Если условия на границе и выражения для инвариантов линейно зависят от компонент $\vec{u}$, то для корректности 
достаточно чтобы СЛАУ была разрешима.




\end{document}
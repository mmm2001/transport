\documentclass[../../main/main.tex]{subfiles}

\begin{document}

\section{Мультипликативные методы ускорения сходимости итераций по рассеянию}

\subsection{Метод квазидиффузии Гольдина В.Я.}

Метод квазидиффузии (КД), предложенный В.Я. Гольдиным, представляет собой итерационный подход 
к решению стационарного уравнения переноса частиц:
\[
\mu \frac{\partial \Psi(x,\mu)}{\partial x} + \Sigma_t(x) \Psi(x,\mu) = \frac{\Sigma_s(x)}{2} \Phi(x) + \frac{Q(x)}{2},
\]
$$ \mu>0 \quad \psi(0, \mu)=\psi^{in}=\psi^0  $$
$$ \mu<0 \quad \psi(X,\mu)=\psi(X,-\mu)$$

где $\Psi(x,\mu)$ — угловое распределение потока нейтронов, $\Phi(x) = \int_{-1}^{1} \Psi(x,\mu)\, d\mu$ — скалярный поток, 
а $J(x) = \int_{-1}^{1} \mu \Psi(x,\mu)\, d\mu$ — векторный поток.


Проинтегрируем исходное уравнение с весом 1 по $\mu$ на [-1,1]:

\begin{align*}
\pderiv{}{x} \int_{-1}^{1}\mu \Psi(x, \mu)d\mu + \Sigma_t(x)\int_{-1}^{1}\psi(x,\mu)d\mu =
\frac{\Sigma_s(x)}{2}\Phi(x)\int_{-1}^{1}d\mu+\frac{Q(x)}{2}\int_{-1}^{1}d\mu
\end{align*}

Получаем первое уравнение квазидиффузии:
\[
\pderiv{J(x)}{x} + \Sigma_a(x)\Phi(x)=Q(x)
\]

Проинтегрируем исходное уравнение с весом $\mu$ по $\mu$ на $[-1,1]$:
\begin{align*}
\frac{d}{dx} \int_{-1}^{1} \mu^2 \Psi(x, \mu) \, d\mu &+ \Sigma_t(x) \int_{-1}^{1} \mu \Psi(x, \mu) \, d\mu \\
&= \frac{\Sigma_s(x)}{2} \Phi(x) \int_{-1}^{1} \mu \, d\mu + \frac{Q(x)}{2} \int_{-1}^{1} \mu \, d\mu
\end{align*}

Учитывая, что $\int_{-1}^{1} \mu \, d\mu = 0$, получаем второе уравнение квазидиффузии:
\[
\pderiv{}{x}\bigl(D(x)\Phi(x)\bigr) + \Sigma_t(x) J(x) = 0
\]
где $J(x) = \int_{-1}^{1} \mu \Psi(x, \mu) \, d\mu$ --- ток нейтронов.

Основная идея метода состоит в замене исходного интегро-дифференциального уравнения 
системой двух связанных дифференциальных уравнений:
\[
\begin{cases}
\ \displaystyle \pderiv{J(x)}{x} + \Sigma_a(x) \Phi(x) = Q(x) \\[8pt]
\ \displaystyle \pderiv{}{x}\bigl(D(x)\Phi(x)\bigr) + \Sigma_t(x) J(x) = 0
\end{cases}
\]

\[
D(x) = \frac{\int_{-1}^{1} \mu^2 \Psi(x,\mu)\, d\mu}{\int_{-1}^{1} \Psi(x,\mu)\, d\mu}
\]
--- \textit{фактор Эддингтона}, играющий роль коэффициента квазидиффузии.
Для $P_1$ приближения коэффциент равен $\frac 13$ \\

При этом условии имеем следующую систему квазидиффузии:


\[
\begin{cases}
\ \displaystyle \pderiv{J(x)}{x} + \Sigma_a(x) \Phi(x) = Q(x) \\[8pt]
\ \displaystyle \pderiv{}{x} \Bigl(\frac{1}{\Sigma_t} \pderiv{D(x)\Phi(x)}{x}\Bigr)+
        \Sigma_a(x) \Phi(x)=Q(x)
\end{cases}
\] 

\begin{itemize}
    \item \textbf{Граничные условия на левой границе}:
    \[
    \alpha = \frac{J(0) - J^{\mathrm{in}}}{\Phi(0) - \Phi^{\mathrm{in}}},
    \]
    где 
    \[
    \Phi^{\mathrm{in}} = \int_{0}^{1} \Psi^{\mathrm{in}}(\mu) \, d\mu,
    \quad
    J^{\mathrm{in}} = T^{\mathrm{in}} = \int_{0}^{1} \mu \Psi^{\mathrm{in}}(\mu) \, d\mu,
    \]
    а $\Psi^{\mathrm{in}}(\mu)$ задаёт входящий поток при $x=0$.

    \item \textbf{На правой границе} принимается условие свободного выхода:
    \[
    J(X) = 0.
    \]
\end{itemize}

Если известны заданы начальные приближения для тока и скалярного потока то по $P_1$ приближение
для коэффциентов $D^{(0)}(x)=\frac 13 \quad \alpha^{(0)}(0)=- \frac 12 $\\




\subsection{Алгоритм метода квазидиффузии}
Итерационный процесс метода квазидиффузии состоит из следующих шагов:
\begin{enumerate}
    \item \textit{LO-шаг} Решаются уравнения квазидиффузии c граничными условиями:
\[
\begin{cases}
\ \displaystyle \pderiv{J(x)^{l+\frac 12}}{x} + \Sigma_a(x) \Phi(x)^{l+\frac 12} = Q(x) \\[8pt]
\ \displaystyle \pderiv{}{x}\bigl(D^{l}(x)\Phi^{l+\frac 12}(x)\bigr) + \Sigma_t(x) J^{l+\frac 12}(x) = 0
\end{cases}
\]
получаем значения для скалярного потока и тока на полуслое

    \item \textit{HO-шаг} Решение уравнения переноса с исходными граничными условиями:
\[
\mu \frac{\partial \Psi^{l+1}(x,\mu)}{\partial x} + \Sigma_t(x) \Psi^{l+1}(x,\mu) = \frac{\Sigma_s(x)}{2} \Phi^{l+\frac 12}(x) + \frac{Q(x)}{2},
\]
Из решения на этом этапе находим значения коэффицентов $D, \quad \alpha$ на новом слое 
\end{enumerate}



Существуют обобщения метода квазидиффузии, такие как \textit{WA-метод} (Weighted-Alpha), 
в котором вместо одного коэффициента $\alpha$ используются весовые моменты с показателем $\alpha \geq 0$:
\[
A_{\pm}^{(l+1/2)}(x) = (\alpha+1) \frac{\int_{0}^{1} \mu^{\alpha} \Psi_{\pm}^{(l+1/2)}(x,\mu)\, d\mu}
{\int_{0}^{1} \Psi_{\pm}^{(l+1/2)}(x,\mu)\, d\mu},
\]
\[
B_{\pm}^{(l+1/2)}(x) = (\alpha) \frac{\int_{0}^{1} \mu^{\alpha+1} \Psi_{\pm}^{(l+1/2)}(x,\mu)\, d\mu}
{\int_{0}^{1} \Psi_{\pm}^{(l+1/2)}(x,\mu)\, d\mu}.
\]
Оптимизация параметра $\alpha$ (например, $\alpha \approx 0{,}366$) позволяет существенно улучшить 
скорость сходимости по сравнению с классическим методом КД.

Метод квазидиффузии Гольдина показывает высокую эффективность при решении задач переноса 
с сильным рассеянием и является одним из наиболее мощных мультипликативных методов 
ускорения итераций по рассеянию.


\begin{figure}[ht]
    \centering
    \includegraphics[width=0.8\linewidth]{3.6_pic1.png}
\end{figure}

\end{document}
\documentclass[../../main/main.tex]{subfiles} 

\begin{document}
\section{HOLO алгоритмы решения уравнений переноса}

\subsection*{Основная концепция}
HOLO (Higher order - Low order) алгоритмы --- это семейство итерационных методов для численного решения уравнений переноса частиц, которые комбинируют:
\begin{itemize}
    \item \textbf{High-Order, HO} описание углового распределения частиц
    \item \textbf{Low-Order, LO} приближение для интегральных характеристик (скалярный поток, векторный поток)
\end{itemize}

\subsection*{Принцип работы}
Алгоритм строится на итерационном процессе, где две системы уравнений решаются последовательно, обмениваясь информацией:

\subsubsection*{1. High-Order система}
Полное угловое уравнение переноса в дискретном виде (например метод иттераций источника):

\subsubsection*{2. Low-Order система}
Упрощённые моментные уравнения (диффузионное приближение или $P_1$-система):


\subsection*{Базовый алгоритм}
\begin{enumerate}
    \item Инициализация: задаём начальное приближение $\phi^{(0)}$
    
    \item Для каждой итерации $k = 0,1,2,\dots$:
    \begin{enumerate}
        \item \textbf{HO-шаг}: Решаем уравнение переноса с фиксированным скалярным потоком
        Вычисляем угловые моменты
        
        \item \textbf{LO-шаг}: Решаем моментные уравнения с обновлёнными коэффициентами
        
        \item Обновляем факторы связи:
        Коэффциент квазидиффузии
    \end{enumerate}
    
    \item Проверяем сходимость:
\end{enumerate}

\subsection*{Основные типы HOLO-методов}

\subsubsection*{Аддитивные алгоритмы:}
\begin{enumerate}
    \item DSA (Diffusion Synthetic Acceleration) ускорение
    \item $S_2$\&A ($S_2$ Synthetic Acceleration) ускорение
    \item T\&A (Transport \& Acceleration) ускорение
    \item KP-методы (Krylov Projection) Лебедева
\end{enumerate}

\subsubsection*{Мультипликативные алгоритмы:}
\begin{enumerate}
    \item Метод квазидиффузии (Quasidiffusion) Гольдина В.Я.
    \item AWM (Angular Weighted Method)
\end{enumerate}

\end{document}
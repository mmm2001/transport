\documentclass[../../main/main.tex]{subfiles} 

\begin{document}

\section{Кубатуры на сфере для анизотропного рассеяния как прямое произведение сеток по каждому
из углов отдельно}

Угловое направление движения частицы $\Omega$ в сферической системе координат
на сфере задается двумя углами $\theta, \quad \phi$ - полярный и азимутальный углы\\
Полная угловая сетка строится как прямое произведение одномерных сеток по пространственным углам. \\
Веса кубатур вычисляются как: $\omega_m = \omega_{ij} = \omega_{\mu_i} \cdot \omega_{\phi_j}$
\\
\\
\begin{itemize}
    \item Для посторения сетки по $\phi$ используется $2N$ равностоящих узлов
    \item Для построения сетки по $\mu$ в качестве узлов выбираются нули многочленов Лежандра.
          Такая квадратнурая формула точна для многочленво степени $2N-1$
\end{itemize}
Всего имеем сетку состоящую из   $2N^2$ узлов


\begin{figure}[ht]
    \centering
    \begin{minipage}[t]{0.4\textwidth}
        \centering
        \includegraphics[width=\linewidth]{5.4_pic1.png}
    \end{minipage}
    \hfill
    \begin{minipage}[t]{0.4\textwidth}
        \centering
        \includegraphics[width=\linewidth]{5.4_pic2.png}
    \end{minipage}
\end{figure}


\end{document}
\documentclass[../../main/main.tex]{subfiles} 

\begin{document}

\section{Уравнение переноса в форме уравнения второго порядка для четной, нечетной компонент решения по угловой переменной и для исходного углового потока}

\textit{Что дает использование уравнения второго порядка?}\\
Оператор второго порядка (типа эллиптического) часто является симметричным и положительно определенным.
Естественная свзяь с уравнением для поправок DSA метода.
\\
Рассмотрим уравнение переноса в одномерной плоской геометрии:
\begin{equation*}
\pderiv{\Psi(x,\mu)}{x} + \Sigma_t(x) \Psi(x,\mu) = \frac{\Sigma_s(x)}{2} \Phi(x) + \frac{Q(x)}{2},
\end{equation*}
где
\begin{equation*}
\Phi(x) = \int_{-1}^{1} \Psi(x,\mu') \, d\mu' \quad\text{ - скалярный поток}
\end{equation*}

Граничные условия задаются следующим образом:
\begin{align*}
\Psi(0,\mu) &= \Psi^{\text{in}}(\mu), \quad \mu > 0, \\
\Psi(X,\mu) &= \Psi(X,-\mu), \quad \mu < 0.
\end{align*}

Введём чётную и нечётную компоненты углового потока:
\[
\begin{cases}
\Psi^{+}(x,\mu) &= \frac{1}{2} \left( \Psi(x,\mu) + \Psi(x,-\mu) \right) \\
\Psi^{-}(x,\mu) &= \frac{1}{2} \left( \Psi(x,\mu) - \Psi(x,-\mu) \right)
\end{cases}
\]

так что

\begin{equation*}
\Psi(x,\mu) = \Psi^{+}(x,\mu) + \Psi^{-}(x,\mu).
\end{equation*}
\vspace{5 px}

Система уравнений для $\Psi(x,\mu)$ и $\Psi(x,-\mu)$:
\[
\begin{cases}
\pderiv{\Psi(x,\mu)}{x} + \Sigma_t(x) \Psi(x,\mu) &= \frac{\Sigma_s(x)}{2} \Phi(x) + \frac{Q(x)}{2}, \\
\pderiv{\Psi(x,-\mu)}{x} + \Sigma_t(x) \Psi(x,-\mu) &= \frac{\Sigma_s(x)}{2} \Phi(x) + \frac{Q(x)}{2}.
\end{cases}
\]


Складывая и вычитая эти уравнения, получаем уравнения для чётной и нечётной компонент:

\[
\begin{cases}
\pderiv{\Psi^{+}(x,\mu)}{x} + \Sigma_t(x) \Psi^{+}(x,\mu) &= \frac{\Sigma_s(x)}{2} \Phi(x) + \frac{Q(x)}{2}, \\
\pderiv{\Psi^{-}(x,\mu)}{x} + \Sigma_t(x) \Psi^{-}(x,\mu) &= 0.
\end{cases}
\]

Полный скалярный поток выражается через чётную компоненту:
\begin{align*}
\Phi(x) &= \int_{-1}^{1} \Psi(x,\mu) \, d\mu 
= \int_{-1}^{1} \left( \Psi^{+}(x,\mu) + \Psi^{-}(x,\mu) \right) d\mu 
= \int_{-1}^{1} \Psi^{+}(x,\mu) \, d\mu \nonumber \\
&= 2 \int_{0}^{1} \Psi^{+}(x,\mu) \, d\mu
\end{align*}

1. Из уравнения для нечётной компоненты следует:
\begin{equation*}
\Psi^{-}(x,\mu) = -\frac{\mu}{\Sigma_t(x)} \pderiv{\Psi^{+}(x,\mu)}{x}.
\end{equation*}

Подставляя это выражение в уравнение для чётной компоненты, получаем уравнение второго порядка:
\begin{equation*}
- \mu^2 \pderiv{}{x} \left( \frac{1}{\Sigma_t(x)} \pderiv{\Psi^{+}(x,\mu)}{x} \right) + \Sigma_t(x) \Psi^{+}(x,\mu) = \frac{\Sigma_s(x)}{2} \Phi(x) + \frac{Q(x)}{2}.
\end{equation*}

\begin{itemize}
    \item Это уравнение имеет форму диффузионного уравнения, используемого в методе DSA (Diffusion Synthetic Acceleration).
    \item Однако постановка граничных условий для этого уравнения затруднена.
\end{itemize}




2. Из первого уравнения можно выразить $\Psi^{+}(x,\mu)$:
\begin{equation*}
\Psi^{+}(x,\mu) = -\frac{\mu}{\Sigma_t(x)} \pderiv{\Psi^{-}(x,\mu)}{x} + \frac{\Sigma_s(x)}{2 \Sigma_t(x)} \Phi(x) + \frac{Q(x)}{2 \Sigma_t(x)},
\end{equation*}
что приводит к другому уравнению второго порядка для $\Psi^{-}(x,\mu)$:
\begin{equation*}
- \mu^2 \pderiv{}{x} \left( \frac{1}{\Sigma_t(x)} \pderiv{\Psi^{+}(x,\mu)}{x} \right) + \Sigma_t(x) \Psi^{-}(x,\mu) = -\mu \pderiv{}{x} \left( \frac{\Sigma_s(x)}{2} \Phi(x) + \frac{Q(x)}{2} \right).
\end{equation*}


Для этого уравнения также трудно сформулировать общие граничные условия, поскольку они естественным образом задаются для $\Psi$, а не для $\Psi^{+}$ или $\Psi^{-}$



3.Рассмотрим самосопряжённое уравнение для углового потока (SAAP — Self-Adjoint Angular Flux equation):
\begin{align*}
- \mu^2 \pderiv{}{x} \left( \frac{1}{\Sigma_t(x)} \pderiv{\Psi(x,\mu)}{x} \right) + \Sigma_t(x) \Psi(x,\mu) = \\
= \frac{\Sigma_s(x)}{2} \Phi(x) + \frac{Q(x)}{2} + \mu \pderiv{}{x} \left( \frac{\Sigma_s(x)}{2} \Phi(x) + \frac{Q(x)}{2} \right).
\end{align*}
Для этого уравнения граничные условия формулируются естественным образом. Однако при переходе к многомерным задачам (включая зависимость от энергии) могут возникнуть дополнительные сложности.

\end{document}
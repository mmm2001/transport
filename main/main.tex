
\documentclass[12pt, a4paper]{article}


\usepackage{subfiles}


\usepackage[utf8]{inputenc}
\usepackage[T2A]{fontenc}
\usepackage[english, russian]{babel}
\usepackage{amsmath, amssymb, amsthm}
\usepackage{graphicx}
\usepackage{hyperref}
\usepackage{geometry}
\usepackage{graphics}
\geometry{left=2.5cm, right=2.5cm, top=2.5cm, bottom=2.5cm}
\graphicspath{ {../images/} }

\title{Уравнения переноса: ответы на вопросы}
\author{НЕМ}
\date{\today}


\DeclareMathOperator{\En}{E_n} 
\newcommand{\pderiv}[2]{\frac{\partial #1}{\partial #2}}
\newcommand{\deriv}[2]{\frac{d #1}{d #2}}


\begin{document}


\maketitle
\newpage

\tableofcontents
\newpage 


% ============ ПОДКЛЮЧЕНИЕ ПОДФАЙЛОВ ============
\subfile{../N/1.1/1.1.tex}
\pagebreak
\subfile{../E/1.2/1.2.tex}
\pagebreak
\subfile{../M/1.3/1.3.tex}
\pagebreak
\subfile{../N/1.4/1.4.tex}
\pagebreak
\subfile{../N/2.1/2.1.tex}
\pagebreak
\subfile{../E/2.2/2.2.tex}
\pagebreak
\subfile{../M/2.5/2.5.tex}
\pagebreak
\subfile{../E/3.2/3.2.tex}
\pagebreak
\subfile{../M/3.3/3.3.tex}  %вообще не имею понятия что задесь писать, будто бы Егор написал все что у меня здесь есть
%в своем вопросе. Вообще он как то странно сформулирован. Ху%ня какая то 
\pagebreak
\subfile{../N/3.4/3.4.tex}
\pagebreak
\subfile{../E/3.5/3.5.tex}
\pagebreak
\subfile{../E/3.8/3.8.tex}
\pagebreak
\subfile{../M/3.6/3.6.tex}
\pagebreak
\subfile{../N/4.1/4.1.tex}
\pagebreak
\subfile{../E/4.2/4.2.tex}
\pagebreak
\subfile{../N/5.2/5.2.tex}
\pagebreak
\subfile{../N/6.2/6.2.tex}
\pagebreak
\end{document}

\documentclass[../../main/main.tex]{subfiles}

\begin{document}
\section{Поглощение и излучение в задачах радиационного переноса. Аналогия с задачами переноса нейтронов}

Рассматривается система уравнений высокотемпературной радиационной газовой динамики.

Выделяют блок совместного решения уравнения переноса с уравнением энергии (условно, при замороженной $\operatorname{div} \vec{w}_T$, $\vec{w}_T$ -- тепловой поток).

Многогрупповое приближение переноса излучения.

\begin{equation*}
\frac{1}{c} \frac{\partial \psi_g(x,\mu,t) }{\partial t}
+ \mu \frac{\partial \psi_g(x,\mu,t) }{\partial x}
+ \Sigma_g (x,t) \psi_g(x,\mu,t)=
\Sigma_g (x,t) B_g(x,t)
\end{equation*}
$\Sigma_g$ -- коэффициент полгощения. $B_g$ -- Планковская функция собственного излучения плазмы.
Уравнение энергии:
\begin{equation*}
c_p \frac{\partial T (x,t)}{\partial t} = \sum_{g=1}^{G} \Sigma_g (x,t) \int_{-1}^{1} \left[ \psi_g(x, \mu, t) - B_g(x,\mu,t) \right] d\mu + Q(x)
\end{equation*}

Функция Планка:
\begin{equation*}
B_g(T) = \int_{\nu_g - 1/2}^{\nu_g + 1/2} \frac{4 \pi h \nu^3}{c^2} \frac{1}{c^{(h \nu) / (kT)}} d\nu
\end{equation*}
$y:= h\nu / kT$
\begin{equation*}
B_g(T) = \int_{y_g - 1/2}^{y_g + 1/2} \frac{4 \pi h y^3 \left( \frac{kT}{h} \right)^3 }{c^2} \frac{1}{e^y-1} \frac{kT}{h} dy = \sigma_g T^4 
\end{equation*}
\begin{equation*}
\sigma_g = \frac{k^4}{h^4} \frac{4 \pi}{c^2} \int_{y_g - 1/2}^{y_g+1/2} \frac{1}{e^y-1}dy
\end{equation*}

Явная схема бессмысленна -- $\frac{c \tau}{h} < 1$, но $c$ -- скорость света!

Используем неявную аппроксимацию по времени.

Обозначения:
$\psi (x, \mu, t^{n+1}) \rightarrow \psi(x,\mu)$ \\
$\psi (x, \mu, t^n) \rightarrow \psi^n(x,\mu)$ \\
Индекс есть -- старый шаг, индекса нет -- новый шаг.

\begin{equation*}
\frac{1}{c} \frac{\psi_g(x\mu) - \psi_g^n (x,\mu) }{\Delta t}
+ \mu \frac{\partial \psi_g (x,\mu)}{\partial x}
+ \Sigma_g(x) \psi_g(x,\mu) = \Sigma_g(x) \tilde{B}_g(x)
\end{equation*}

\begin{equation*}
c_p \frac{ T(x) - T^n(x) }{\Delta t} =
\sum_{g=1}^G \Sigma_g \int_{-1}^{1} \left[ \psi_g (x,\mu) - \tilde{B}_g(x) \right] d \mu + Q(x)
\end{equation*}

\begin{equation*}
\tilde{B}_g(x) = B_g^n(x) + \left[ T(x) - T^n(x) \right] \dot{B}_g(x)
\end{equation*}
$\dot{B}_g$ -- производная по времени.

На картинке есть какие-то пояснения, но я не знаю, что они значат.
\begin{figure}[h!]
\centering
\includegraphics[width = 0.9\textwidth]{4.7_pic1.jpg}
\end{figure}

Исключая эту разность $T(x) - T^n(x)$,

\begin{equation*}
\mu \frac{\partial \psi_g (x,\mu) }{\partial x}
+ \tilde{\Sigma}_g (x) \psi_g(x,\mu)
=
\frac{\chi_g}{2} \sum_{g=1}^G \eta(x) \Sigma_g(x)
\int_{-1}^{1} \psi_g(x,\mu) d\mu + S_g(x,\mu)
\end{equation*}
где:
\begin{itemize}
\item $\tilde{\Sigma}_g(x) = \Sigma_g(x)  \frac{1}{c \delta t}$
\item $\eta (x) = 
\frac{\sum_{g=1}^G \Sigma_g(x) \dot{B}_g^n(x)}
{c_p/(2\Delta t) + \sum_{g=1}^G \Sigma_g(x) \dot{B}_g^n(x) }$
\item $\chi_g(x) = 
\frac{\Sigma_g(x) \dot{B}_g(x)}
{\sum_{g'=1}^G \Sigma_{g'}(x) \dot{B}_{g'} (x)}$
\item
\begin{equation*}
S_g(x,\mu) = \frac{1}{c \Delta t} \psi_g^n (x, \mu) + \Sigma_g B_g^n (x) +
\eta \frac{\chi_g}{2} \left[ Q(x) + 2 \sum_{g'=1}^G \Sigma_{g'} (x) B_{g'}^n (x) \right]
\end{equation*}

\end{itemize}

Аналогия с подкритичными задачами переноса нейтронов. Требует ускорения итераций по "рассеянию", возникшему из уравнения энергии.

Аналогия здесь по всей видимости в том, что полученное уравнение на $\psi_g$ имеет такой же вид, как рассматриваемое нами ранее уравнение для переноса нейтронов $\rghitarrow$ к нему применимы все те же методы решения.

\begin{figure}[h!]
\centering
\includegraphics[width = 0.9\textwidth]{4.7_pic2.png}
\end{figure}


\end{document}

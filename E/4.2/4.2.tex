\documentclass[../../main/main.tex]{subfiles}

\begin{document}
\section{Метод итераций по делению (PI-метод). В чем его главная трудность?}

\subsection{Проблема нахождения собственных значений}

Рассматривается уравенение:

\begin{equation*}
\mu \frac{\partial \psi (x,\mu)}{\partial x} + \Sigma_t(x) \psi (x,\mu) = \frac{\Sigma_s(x)}{2} \int_{-1}^{1} \psi (x, \mu') d\mu' + \underbrace{  \frac{1}{k} \frac{1}{2} \nu \Sigma_f(x) \int_{-1}^{1} \psi (x, \mu') d\mu'  }_{\text{член деления}}
\end{equation*}
$k$ -- коэффициент сжатия нейтронов.
\begin{itemize}
\item $k=1$ -- критический режим (поддерживается в реакторе)
\item $k<1$ -- подкритический режим (затухание)
\item $k>1$ -- бесконтрольное увеличение (ЯДЕГХКА САМОСИР)
\end{itemize}
Ставится, как и ранее, ГУ:
\begin{itemize}
\item $\psi(0,\mu) = 0$, $\mu >0$
\item $\psi(X, \mu) = \psi (X, -\mu)$, $\mu < 0$
\end{itemize}

Требуется найти такое $k$, при котором существует нетривиальное решение уравнения:
\begin{equation*}
L \psi(x,\mu) = \frac{1}{k} P \psi(x, \mu)
\end{equation*}
где $L$, $P$ -- линейные операторы:
\begin{equation*}
L \psi(x,\mu) = \mu \frac{\partial \psi (x,\mu)}{\partial x} + \Sigma_t(x) \psi (x,\mu) - \frac{\Sigma_s(x)}{2} \int_{-1}^{1} \psi (x, \mu') d\mu'
\end{equation*}
\begin{equation*}
P \psi(x,\mu) = \frac{1}{k} \frac{1}{2} \nu \Sigma_f(x) \int_{-1}^{1} \psi (x, \mu') d\mu'
\end{equation*}
Получаем:
\begin{equation*}
k \psi(x,\mu) = A \psi (x,\ mu)
\end{equation*}
$A = L^{-1}P$ -- ограниченный положительно определенный оператор (и, соответственно, матрица при дискретизации).

\subsection{Power iteration (метод итераций по делению, степенной метод)}
Внешние итерации по члену деления.
\begin{equation*}
L \psi^{(n+1/2)} (x,\mu) = P\psi^{(n)} (x,\mu)
\end{equation*}
$\psi^{(0)} (x,\mu) \neq 0$.
Нахождение $\psi^{(n+1/2)} (x,\mu)$ означает итерации по рассеянию с известным источником $Q(x)/2 = P \psi^{(n)} (x,\mu)$.

\begin{equation*}
\psi^{(n+1)} (x,\mu) = \frac{\psi^{(n+1/2)} (x,\mu)}{ \| \psi^{(n+1/2)} (x,\mu) \|}
\end{equation*}

\begin{equation*}
k^{(n+1)} = \| \psi^{(n+1/2)} (x,\mu) \|
\end{equation*}

\begin{equation*}
k_1^{(n)} = \frac{\| A^n \psi^0 (x,\mu) \| }{\| A^{n-1} \psi^0 (x,\mu) \|}
\end{equation*}

Обозначим $u_n$ -- собственные функции оператора $A$.

\begin{equation*}
\psi^{(0)} = \sum_{m=1}^\infty a_m u_m(x,\mu)
\end{equation*}
-- разложение по базису из собственных функций. \\

\begin{equation*}
\psi^{(n)} = \frac{A^n \psi^{(0)} }{\| A^n \psi^{(0)} \|} = \frac{\sum_{m=1}^{\infty} a_m k_m^n u_m(x,\mu) }{\| \sum_{m=1}^{\infty} a_m k_m^n u_m(x,\mu) \|}
\end{equation*}

\begin{align*}
\psi^{(1)} &= \frac{  a_1 k_1^n u_1 +\sum_{m=2}^{\infty} a_m k_m^n u_m(x,\mu) }{ \| a_1 k_1^n u_1 +\sum_{m=2}^{\infty} a_m k_m^n u_m(x,\mu) \|} \\
&= \frac{a_1u_1 + \sum_{m=2}^\infty a_m \left( \frac{k_m}{k_1} \right)^n u_m(x,\mu) }{ \| a_1u_1 + \sum_{m=2}^\infty a_m \left( \frac{k_m}{k_1} \right)^n u_m(x,\mu) \|} \\
&= u_1 + O\left( \left( \frac{k_2}{k_1} \right) \right)
\end{align*}

\begin{equation*}
k_1^{(n)} = \frac{\| A^n \psi^{(0)} (x,\mu) \|}{\| A^{n-1} \psi^{(0)} (x,\mu) \|} = k_1 + O\left( \left( \frac{k_2}{k_1} \right) \right)
\end{equation*}

Скорость сходимости итераций по делению определяется $r = k_2/k_1$. Чем ближе $r$ к $1$, тем хуже сходимость; $r \rightarrow 1$ -- сходимость отсутствует. Число PI, требуемое для достижения заданной точности -- $N$. Число расчет уравнения переноса (транспортный проход) есть $N \cdot$ среднее число итераций по рассеянию: в $N$ раз дороже решения задачи на рассеяние.

\subsubsection*{Степенной метод: альтернативное изложение}

А теперь нормальное изложение, не так путанно как в лекциях.

Имеем задачу на собственные значения:

\begin{equation*}
k \psi(x,\mu) = A \psi (x,\mu)
\end{equation*}

Будем применять \textbf{степенной метод}. Он имеет вид:
\begin{enumerate}
\item $f^{(0)}$ -- начальная собственная функция.
\item Итерации: 
\begin{equation*}
f_k = Ag_{k-1}
\end{equation*}
$g_{k-1} = f_{k-1} / \| f_{k-1} \|$.
\end{enumerate}

У $A$ существует базис в $\mathbb{R}^n$ из собственных функций. Тогда:
\begin{equation*}
y_0 = \alpha_1 u_1 + \alpha_2 u_2 + \dots
\end{equation*}
Соответственно,
\begin{equation*}
A^n y_0 = k_1^n \alpha_1 u_1 + k_2^n \alpha_2 u_2 + \dots
\end{equation*}
Рассмотрим $\frac{\|  A^n f_0  \|}{\| A^{n-1} f_0  \|}$:

\begin{align*}
\frac{\| A^n f_0 \|}{\| A^{n-1} f_0 \|} &= \frac{\| k_1^n \alpha_1 u_1 + k_2^n \alpha_2 u_2 + \dots \|}{\| k_1^{n-1} \alpha_1 u_1 + k_2^{n-1} \alpha_2 u_2 + \dots \|} = 
k_1 \frac{\| k_1^n \alpha_1 u_1 + k_2^n \alpha_2 u_2 + \dots \|}{\| k_1^n \alpha_1 u_1 + k_2^{n-1} k_1 \alpha_2 u_2 + \dots \|} \\
&= k_1 \frac{k_1^n \alpha_1}{k_1^n \alpha_1}
\frac{\| u_1 + \frac{k_1^n \alpha_2}{k_1^n \alpha_1}u_2 + \dots \|}{\| u_1 + k_2^{n-1}\frac{k_1}{k_1^n \alpha_1}u_2 + \dots \|} = 
k_1 \frac{\| u_1 + O(\left| \frac{k_2}{k_1} \right|^n)u_2 + \dots \|}{\| u_1 + O(\left| \frac{k_2}{k_1} \right|^{n-1})u_2 + \dots \|}
\end{align*}

Т.е, 
\begin{equation*}
\frac{\| A^n f_0 \|}{\| A^{n-1} f_0 \|} \rightarrow k_1
\end{equation*}

Отсюда также видно, что $k_2/k_1$ оказывает ключевое влияние на сходимость -- чем меньше эта величина, тем лучше сходимость степенного метода.

\end{document}

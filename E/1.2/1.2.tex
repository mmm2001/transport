\documentclass[../../main.tex]{subfiles}

\begin{document}
\graphicspath{ {./images/} }

\section{Методы построения схем для численного решения уравнения переноса: конечно-разностный,
метод неопределенных коэффициентов, интерполяционно-характеристический, метод прямых.
Примеры каждого из методов. Вывод схемы Лакса-Вендроффа для линейного уравнения
переноса из вычитания главного члена погрешности аппроксимации явной четырехточечной
схемы.}

Будем рассматривать уравнение переноса в некоторой области:

\begin{equation*}
\frac{\partial u}{\partial t} + a \frac{\partial u}{\partial x} = 0
\end{equation*}

Введем сетку из $M$ узлов по пространству и $N$ по времени, $h$, $\tau$ -- соответствующие шаги.

\subsection{Метод неопределенных коэффициентов}
Любую разностную схему на некотором шаблоне можно представить в виде:
\begin{equation*}
\sum a_{k}y_{m}^{n} = 0
\end{equation*}
где $m$,$n$ принадлежат шаблону, $k$ -- номер коэффициента.

Тогда для получения разностной схемы необходимо подставить в разностный оператор разложения $u$ в ряд Тейлора и приравнять коэффициенты так, чтобы разностный оператор давал исходную дифференциальную задачу. Приведем пример: вывод схемы Лакса-Вендроффа.

\subsubsection*{Схема Лакса-Вендроффа: МНК}

\begin{figure}[h!]
\includegraphics[width = 0.2\textwidth]{1.2_pic1.png}
\end{figure}

Запишем разностную схему:

\begin{equation*}
ky_n^{n+1}+ly_{m+1}^n+py_n^m+qy_{m-1}^n=0
\end{equation*}
Используем следствие дифференциальной задачи: $\frac{\partial^{(k)}}{\partial t^{(k)}}u = (-c)^k\frac{\partial^{(k)}}{\partial x^{(k)}}u$
\begin{equation*}
k(u_n^m + \dot{u} \tau + \underbrace{ \ddot{u}\tau^2/2 }_{u''c\tau^2/2}  + O(\tau^3) )+l(u_n^m + u' h + u'h^2/2 + O(h^3))+pu_n^m+q(u_n^m - u' h + u'h^2/2 + O(h^3))= \dot{u} + cu'
\end{equation*}
Получаем:
\begin{align*}
u_m^n ~ : ~ & k+l+p+q=1 \\
\dot{u} ~ : ~ & k\tau=1 \\
u' ~ : ~ & lh-qh=c \\
u'' ~ : ~ & kc^2\tau^2/2 + lh^2/2 + qh^2/2 = 0
\end{align*}
Откуда имеем систему:
\[
\begin{cases}
1/\tau +l+p+q=0 \\
lh-qh=c \\
c^2 \tau + lh^2 + qh^2 = 0
\end{cases}
\]

Откуда $l = \frac{c}{2h} - \frac{c^2 \tau}{2h^2}$, $q = -\frac{c}{2h} - \frac{c^2 \tau}{2h^2}$, $p = -\frac{1}{\tau} - \frac{c^2 \tau}{h^2}$.
Подставляя в разностную схему, получаем схему Лакса-Вендроффа:
\begin{equation*}
\frac{y_m^{n+1} - y_m^n}{\tau}  + c \frac{y_{m+1}^n - y_{m-1}^n}{2h} - c^2\tau \frac{y_{m+1}^n - 2y_m^n + y_{m-1}^n}{2h^2} = 0
\end{equation*}

\subsection{Интерполяционно-характеристический метод}

Интерполяционно-характеристический метод основан на интерполяции значений на $n$-м временном слое в точку $A$, лежащую на характеристике, опущенной из точки $(m;n+1)$. Таким образом, в точку $(m;n+1)$ решение переносится точно. Интерполяцию в $A$ можно выполнять любым способом, это определяет порядок аппроксимации.

\subsubsection*{Схема Лакса-Вендроффа: интерполяционно-характеристический метод}

\begin{figure}[h!]
\includegraphics[width = 0.2\textwidth]{1.2_pic2.png}
\end{figure}

Разностная схема примет вид: 

\begin{equation*}
y_{m}^{n+1} = P_{m-1}(x_A)y_{m-1}^{n} + P_{m}(x_A)y_{m}^{n} + P_{m+1}(x_A)y_{m+1}^{n} 
\end{equation*}
$x_A = x_m - \sigma h$ -- из уравнения характеристики, $\sigma = c\tau/h$. Для интерполяции используем трехточечные полиномы Лагранжа (что обеспечит второй порядок аппроксимации решения).

\begin{equation*}
P_{m-1} (x) = \frac{(x_{}-x_{m})(x_{}-x_{m+1})}{(x_{m-1}-x_{m})(x_{m-1}-x_{m+1})}
\end{equation*}

\begin{align*}
P_{m-1}(x_A) &= \frac{(x_m - \sigma h-x_{m})(x_m - \sigma h-x_{m+1})}{(x_{m-1}-x_{m})(x_{m-1}-x_{m+1})} \\
& = \frac{(-\sigma h)(-h - \sigma h)}{-h(-2h)} = \frac{1}{2} \sigma (1+\sigma) 
\end{align*}

\begin{equation*}
P_m(x) = \frac{(x-x_{m-1})(x-x_{m+1})}{(x_m-x_{m-1})(x_m-x_{m+1})}
\end{equation*}

\begin{align*}
P_m(x_A) &= \frac{(x_m - \sigma h-x_{m-1})(x_m - \sigma h-x_{m+1})}{(x_m-x_{m-1})(x_m-x_{m+1})} \\
&= \frac{(h-\sigma h)(-h-\sigma h)}{(h)(-h)} = (1-\sigma)(1+\sigma)
\end{align*}

\begin{equation*}
P_{m+1}(x) = \frac{(x-x_{m-1})(x-x_m)}{(x_{m+1}-x_{m-1})(x_{m+1}-x_m)}
\end{equation*}

\begin{align*}
P_{m+1}(x_A) &= \frac{(x_m - \sigma h-x_{m-1})(x_m - \sigma h-x_m)}{(x_{m+1}-x_{m-1})(x_{m+1}-x_m)} \\
& = \frac{(h-\sigma h)(-\sigma h)}{(2h)(h)} = \frac{1}{2}(-\sigma)(1-\sigma)
\end{align*}

Получаем разностную схему:
\begin{equation*}
y_{m}^{n+1} = \frac{1}{2} \sigma (1+\sigma) y_{m-1}^n + (1-\sigma)(1+\sigma) y_m^n + \frac{1}{2}(-\sigma)(1-\sigma) y _{m+1}^n
\end{equation*}

\begin{equation*}
y_{m}^{n+1} = y_m^n - \frac{\sigma}{2} (y_{m+1}^n - y_{m-1}^n) + \frac{\sigma^2}{2}(y_{m+1}^n - 2y_m^n + y_{m-1}^n)
\end{equation*}

\subsection{Метод прямых}

Метод прямых состоит некоторой пространственной дискретизации дифференциального оператора $\frac{\partial u}{\partial x}$ и последующем получении системы ОДУ, которые могут быть интегрированы каким-либо образом: метод Эйлера, Рунге-Кутты нужного порядка и т.п.

Метод прямых обладает большей гибкостью. Соответственно, более вычислительно сложен при использовании более точных методов интегрирования дифференциального уравнения.

\subsubsection*{Схема "явный левый уголок": метод прямых}

Рассмотрим шаблон схемы "явный левый уголок" (точки $(m-1,n)$, $(m,n)$, $(m,n+1)$). Тогда,
\begin{equation*}
\frac{\partial u}{\partial x} \approx F_h = c\frac{y_m^n - y_{m-1}^n}{h} 
\end{equation*}
Тогда получаем обыкновенное дифференциальной уравнение. Начальное условие для него известно из начального условия исходной дифференциальной задачи.
\begin{equation*}
\frac{\partial u}{\partial x} = -F_h
\end{equation*}
Данное обыкновенное дифференциальное уравнение можно решать каким-либо способом. Применяя явный метод Эйлера, получим
\begin{equation*}
y_m^{n+1} = y_m^n - c F_h \tau
\end{equation*}
\begin{equation*}
\frac{y_m^{n+1} - y_m^n}{\tau} + c \frac{y_m^n - y_{m-1}^n}{h} = 0
\end{equation*}
Получили схему "явный левый уголок".

\subsection{Вывод схемы Лакса-Вендроффа из вычитания главного члена погрешности аппроксимации явной четырехточечной схемы}

Рассмотрим четырехточечный шаблон (точки $(m-1,n)$, $(m,n)$,$(m+1,n)$, $(m,n+1)$) (см. картинки выше). Введем разностную схему для \textbf{неоднородного} уравнения переноса с помощью правых разностей:
\begin{equation*}
\frac{y_m^{n+1} - y_m^n}{\tau} + c \frac{y_{m-1}^n - y_{m-1}^n}{2h} = f_m^n
\end{equation*}
Рассмотрим невязку:
\begin{align*}
r_{\tau h} &= \mathcal{L}[u]_{\tau h} - f \\
&= \underbrace{ \dot{u} -c u' -f }_{=0} + \frac{1}{2}\ddot{u} \tau + \frac{c}{6} u''' h^2 + O(\tau^2, h^2) 
\end{align*}
Как видим, порядок аппроксимации первый. Идея: вычесть аппроксимацию члена $\frac{1}{2}\ddot{u} \tau$ для получения первого порядка. В силу того, что по времени точки лишь две, нужно перейти к пространственным производным, чтобы аппроксимировать вторую производную центральной разностью. Для этого используем дифференциальные следствия исходной задачи:

$\ddot{u} = -c\dot{u}' + \dot{f}$, $\dot{u}' = -c u'' + f'$;\\
\begin{equation*}
\ddot{u} = c^2u'' -cf' + \dot{f}
\end{equation*}
Аппроксимируем $u''$ центральной разностью:

\begin{equation*}
u'' \approx \frac{y_{m+1}^n - 2y_m^n + y_{m-1}^n}{h^2}
\end{equation*}
Вычтем $\frac{\tau}{2}u''$, заменив $u''$ аппроксимацией.
Получаем разностную схему:

\begin{equation*}
\frac{y_m^{n+1} - y_m^n}{\tau}  + c \frac{y_{m+1}^n - y_{m-1}^n}{2h} - c^2\tau \frac{y_{m+1}^n - 2y_m^n + y_{m-1}^n}{2h^2} = f_m^n + \frac{c\tau}{2}f' - \frac{\tau}{2}\dot{f}
\end{equation*}
Полученная схема будет иметь порядок аппроксимации $O(\tau^2, h^2) $
\end{document}

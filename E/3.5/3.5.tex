\documentclass[../../main/main.tex]{subfiles}

\begin{document}
\section{Аддитивные методы итераций по рассеянию}

К \textbf{аддитивным} методам относятся:
\begin{itemize}
\item DSA (Diffusion Synthetic Acceleration)-ускорение
\item S$_N$SA
\item TSA
\item KP-методы Лебедева
\end{itemize}

Все эти методы -- HOLO-алгоритмы, от high-order low-order.

\begin{itemize}
\item HO: high order transport equation $\psi (x,\mu)$
\item LO: low order transport equation $\psi (x)$
\end{itemize}

Почему такие названия?

В \textbf{аддитивных} методах LO выписывается для поправок, которые добавляются к решению.

В \textbf{мультипликативных} методах -- более сложная связь:


\begin{equation*}
\text{HO} _{\text{ Коэфф. диффузии+коэффы ГУ} \longrightarrow } ^{ \longleftarrow \text{скалярный поток или или поправл. член рассеяния} } \text{LO}
\end{equation*}


\subsection{KP методы Лебедева и их связь с DSA методами ускорения}

Были созданы как методы расчета ядерных реакторов. В.И. Лебедев, Г.И. Марчук.
Семейство методов:
\begin{equation*}
KP_1(\alpha_1) \dots P_k(\alpha_k)
\end{equation*}

$P_1, \dots, P_k$ -- диффузионные операторы степени $2\alpha_1, \dots, 2\alpha_k$
$K$ -- транспортная часть, т.е. уравнение переноса.

\subsubsection*{Пример: $KP_1(1)$-метод}

\textbf{1 шаг}: решение полномасштабного уравнения переноса (HO-часть)
\begin{equation*}
\mu \frac{\partial \psi^{(l+1/2)}(x, \mu)}{\partial x} + \Sigma_t(x) \psi^{(l+1/2)}(x,\mu) = \frac{\Sigma_s(x)}{2} \Phi^{(l)} + \frac{Q(x)}{2}
\end{equation*}
где $\Phi^{(l)} = \int_{-1}^{1} \psi^{(l)} (x, \mu') d\mu'$.
\\ Краевые условия:
\begin{itemize}
\item $\psi^{(l+1/2)} (0, \mu) = \psi^{in} (\mu)$, $\mu > 0$
\item  $\psi^{(k+1/2)} (X, \mu) = \psi^{(l+1/2)} (X, -\mu)$, $\mu < 0$
\end{itemize}
\textbf{2 шаг}: LO-часть алгоритма
\begin{equation*}
\frac{\partial }{\partial x} \left( \frac{D}{\Sigma_t(x)} \frac{\partial F^{(l+1)}(x)}{\partial x} \right) + \Sigma_a(x) F^{(l+1)} (x) = \Sigma_s(x) (\Phi^{(l+1/2)} (x) - \Phi^{(l)} (x) )
\end{equation*}
где $F$ -- коэффициент при  $P_0 (\mu)$ в разложении $f^{(l)}(x, \mu) = \psi (x, \mu) - \psi^{(l)} (x, \mu)$ по полиномам Лежандра: $f^{(l)}(x, \mu) = \frac{1}{2}\underbrace{P_0 (\mu)}_{=1} F^{(l)} (x) + \frac{3}{2} \underbrace{P_1(\mu)}_{=\mu}G^{(l)}(x)$ (подробнее см. вопрос про DSA-ускорение).
При $D=1/3$ ведет к DSA-методу. Напомним, $D$ -- это \textbf{коэффициент квазидиффузии}; также имеет название  \textbf{фактор Эддингтона}.
\\Краевые условия:
\begin{itemize}
\item $F(0) - \frac{2D}{\Sigma_t(0)}\frac{\partial F(0)}{\partial x} = 0$
\item $\frac{\partial F(X)}{\partial x} = 0$  
\end{itemize}

При $D=0.281$ $\sigma_{KP_1(1)} \leq 0.186C$, а $\sigma_{DSA} \leq 0.2247C$. Т.е. спектральный радиус сходимости меньше чем у DSA.


\subsection{KP$_1$(1)P$_2$(0) метод ускорения Лебедева}

\textbf{1 шаг}: решение полномасштабного уравнения переноса (HO-часть)
\begin{equation*}
\mu \frac{\partial \psi^{(l+1/3)}(x, \mu)}{\partial x} + \Sigma_t(x) \psi^{(l+1/3)}(x,\mu) = \frac{\Sigma_s(x)}{2} \Phi^{(l)} + \frac{Q(x)}{2}
\end{equation*}
где $\Phi^{(l)} = \int_{-1}^{1} \psi^{(l)} (x, \mu') d\mu'$.
\\ Краевые условия:
\begin{itemize}
\item $\psi^{(l+1/2)} (0, \mu) = \psi^{in} (\mu)$, $\mu > 0$
\item  $\psi^{(k+1/2)} (X, \mu) = \psi^{(l+1/2)} (X, -\mu)$, $\mu < 0$
\end{itemize}

\textbf{2 шаг}: $P_1(1)$ -- диффузионное уравнение степени $2 \cdot 1 = 2$.
\begin{equation*}
\frac{\partial }{\partial x} \left( \frac{1/3}{\Sigma_t(x)} \frac{\partial F^{(l+2/3)}(x)}{\partial x} \right) + \Sigma_a(x) F^{(l+2+3)} (x) = \Sigma_s(x) (\Phi^{(l+1/3)} (x) - \Phi^{(l)} (x) )
\end{equation*}

Краевые условия:
\begin{itemize}
\item $F(0) - \frac{2\cdot 1/3}{\Sigma_t(0)}\frac{\partial F(0)}{\partial x} = 0$
\item $\frac{\partial F(X)}{\partial x} = 0$  
\end{itemize}
Условие слева также имеет название \textbf{краевое условие Маршака}.
\vspace{1em} \\
$\Phi^{(l+2/3)} (x) = \Phi^{(l+1/3)} (x) + F^{(l+2/3)} (x)$

\textbf{3 шаг} $P_2(0)$: $LO_2$-часть -- алгебраический пересчет

\begin{equation*}
F^{(l+1)} = \frac{\beta C}{1 - \beta C} \left( \Phi^{(l+2/3)} (x) - \Phi^{(l)} (x) \right)
\end{equation*}
\begin{equation*}
\Phi^{(l+1)} (x) = \Phi^{(l+2/3)} + F^{(l+1)} (x)
\end{equation*}
$C$ -- использовавшееся ранее \textbf{отношение рассеяния}.

\subsection{Фурье-анализ скорости сходимости на примере уравнения переноса в плоскопараллельной геометрии для непрерывного стационарного уравнения переноса }
Фурье-анализ дает:
\begin{align*}
\omega(\lambda) &= \frac{1}{1- \beta C} \left[ \omega_{DSA} (\lambda) -\beta C \right] \\
&=\frac{1}{1-\beta C} \left( \frac{1}{\lambda^2 +3(1-C)} \left[ \left( \frac{\lambda^2}{3} +1 \right) \frac{1}{\lambda} \operatorname{arctg}(\lambda) -1  \right] - \beta \right)
\end{align*}
(напомним, для простых итераций источника: $\omega(\lambda) = \frac{C}{\lambda} \operatorname{arctg}(\lambda)$)

\begin{figure}[h!]
\centering
\includegraphics[width = 0.4\textwidth]{3.5_pic1.jpg}
\end{figure}




\end{document}

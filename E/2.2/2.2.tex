\documentclass[../../main/main.tex]{subfiles}

\begin{document}
\section{Понятие о средней длине пробега и ее связь с макросечением}

Для того, чтобы ввести понятия средней длины пробега и макросечения, разберемся, как описывается реакция, например -- реакция захвата.

Рассматривается квадратная пластинка толщиной, например, 1 атом. В $1$ см$^2$ $N_\alpha$ атомов вещества.
$\sigma$ [см$^2$] -- площадь эффективного сечения, попадая в которое нейтрон взаимодействует с ядром. 

Имеется поток нейтронов плотностью $\nu_0$. 

$\nu_0$ -- количество нейтронов, попадающих на $1$ см$^2$ в $1$ секунду.
$\nu_1$ -- количество нейтронов, вылетающих с обратной стороны пластинки.

Вероятность поглощения (захвата): 
\begin{equation*}
p = \frac{\nu_0-\nu_1}{\nu_0} = N_\alpha ~\sigma
\end{equation*}

\begin{equation*}
\sigma = \frac{1}{N_\alpha}\left(1-\frac{\nu_1}{\nu_0}\right)
\end{equation*}
-- обратное дифференциального сечения захвата.

\begin{figure}[h!]
\includegraphics[width = 0.5\textwidth]{2.2_pic1.png}
\end{figure}

$\rho$ -- плотность вещества.
\begin{equation*}
N = \frac{\rho ~ N_\alpha}{A}
\end{equation*}
$N_\alpha$ -- число Авогадро, $6.022 \cdot 10^{23}$ 1/моль.
измеряется в см$^{-3}$.
\begin{equation*}
N_{\Delta x} = \frac{\rho ~ N_\alpha}{A} \Delta x
\end{equation*}

\begin{equation*}
p(\Delta x) = 1 - \frac{\nu(x+\Delta x)}{\nu(x)} = \underbrace{\sigma N_{\Delta x}}_{\sigma \frac{\rho ~ N_\alpha}{A} \Delta x }
\end{equation*}

\begin{equation*}
\frac{\nu(x) - \nu(x+\Delta x)}{\nu(x)} = \sigma \Delta x \frac{\rho N_\alpha}{A}
\end{equation*}
\begin{equation*}
\frac{\nu (x+ \Delta x) - \nu (x)}{\Delta x} + \sigma \underbrace{ \frac{\rho N_A}{A} }_{=:N} \nu (x)  = 0
\end{equation*}
Устремляя $\Delta x \rightarrow 0$ получаем дифференциальное уравнение:

\begin{equation*}
\nu' + \sigma N \nu = 0
\end{equation*}

Как обычно интегрируем его, получаем

\begin{equation*}
\nu (H) = \nu_0 e^{-\sigma NH}
\end{equation*}

$\Sigma = \sigma N$ -- \textbf{макроскопическое сечение захвата}. Тогда \textbf{расстояние пробега}, т.е. среднее расстояние на котором захватится нейтрон:

\begin{equation*}
l = \frac{1}{\Sigma}
\end{equation*}

Если есть однородная задача с характерными параметрами $H$, то она характеризуется \textbf{оптической толщиной} $\Sigma H$.

\begin{figure}[h!]
\includegraphics[width = 0.5\textwidth]{2.2_pic2.png}
\end{figure}


\end{document}
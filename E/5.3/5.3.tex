\documentclass[../../main/main.tex]{subfiles}

\begin{document}
\section{Кубатуры на сфере для задач переноса в угловом пространстве двух измерений. Изотропное рассеяние. Кубатуры Лебедева.}

\subsection*{Метод дискретных ординат}

Рассматривается изотропное рассеяние: $\omega_0=1$, $\omega_k=0$, $k \geq 1$.

\begin{equation*}
\frac{\Sigma_s}{2} \int \int_{4 \pi} \omega(\mu_0) \psi (\vec{r}, t, \vec{\Omega}' d \vec{\Omega}' = \frac{\Sigma_s}{2}\psi_)(\vec{r}, t)
\end{equation*}

Задача: найти на единичной сфере некоторое количество угловых направлений на единичной сфере. Набор $\vec{\Omega}_l$.

\begin{equation*}
\min l \colon \int \int f(\vec{\Omega}) d \vec{\Omega} = \sum_l c_l f(\vec{\Omega}_l)
\end{equation*}

\subsection*{Выше была информация из лекций. Дальше информация из Дипсика}

Кубатуры Лебедева обеспечивают минимум точек на единичной сфере (тот самый $\min l$).

Исторически сначала использовался такой подход: $S_N$ (квадратуры Гаусса) по $\mu$ и равномерная сетка по $\varphi$ -- те самые дольки апельсина. Но этот вариант имеет недостаток: точки скапливаются у полюсов, что приводит к вычислительным проблемам.

\textbf{Кубатуры Лебедева} решают эти проблемы.

\begin{itemize}
\item У них нет точек на полюсах
\item Покрывают сферу равномерно
\item Для изотропного рассеяния (когда интеграл столкновний сводится к скалярному потоку) их высокая точность обеспечивает корректный расчет.
\end{itemize}

Построение точек кубатуры Лебедева -- нетривиальнач задача:

\begin{enumerate}
\item Используется симметрия октаэдра для уменьшения числа точек-кандидатов
\item Ставятся условия на точность кубатур на полиномах
\item Численно решается полученная система нелинейных уравнений
\end{enumerate}


\end{document}

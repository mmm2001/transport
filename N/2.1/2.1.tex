\documentclass[../../main.tex]{subfiles}

\begin{document}
\section{Основные процессы взаимодействия нейтронов с веществом: перенос, поглощение, рассеяние, деление. Микро- и макросечения. Методы их экспериментального нахождения.}

Основной неизвестной величиной в теории переноса является
\textbf{угловая плотность потока нейтронов}
\[
\Psi(\mathbf r,\boldsymbol{\Omega},E,t),
\]
где:
\begin{itemize}
  \item $\mathbf r=(x,y,z)$ — радиус-вектор точки в пространстве;
  \item $\boldsymbol{\Omega}$ — единичный вектор направления движения нейтрона;
  \item $E$ — кинетическая энергия нейтрона;
  \item $t$ — время;
  \item $\Psi\,d^3r\,d\Omega\,dE$ — число нейтронов,
  находящихся в объёме $d^3r$,
  движущихся в телесном угле $d\Omega$
  с энергией в интервале $[E,E+dE]$.
\end{itemize}

\subsection{Микросечения}

Микросечение $\sigma(E)$ — эффективная площадь взаимодействия
одного нейтрона с одним ядром.

Размерность микросечения:
\[
[\sigma]=\text{см}^2.
\]

Различают:
\begin{itemize}
    \item $\sigma_s(E)$ — рассеяние,
    \item $\sigma_a(E)$ — захват,
    \item $\sigma_f(E)$ — деление,
    \item $\sigma_t(E) = \sigma_a + \sigma_s$.
\end{itemize}

Микросечения резко зависят от энергии и имеют резонансную структуру.

\subsection{Макросечения}

Макросечение определяется формулой:
\[
\Sigma(E)=N\sigma(E),
\]
где $N$ — концентрация ядер (см$^{-3}$).

Макросечения имеют размерность:
\[
[\Sigma]=\text{см}^{-1}.
\]

Средняя длина свободного пробега нейтрона:
\[
l=\frac{1}{\Sigma_t}.
\]

Связь физических процессов и математического описания:
\[
\boxed{
\text{взаимодействия}
\;\Rightarrow\;
\sigma(E)
\;\Rightarrow\;
\Sigma(E)
\;\Rightarrow\;
\text{уравнение переноса}
}
\]

\subsection{Перенос нейтронов}

Перенос — это свободное движение нейтронов между процессами взаимодействия.
Между столкновениями нейтрон движется прямолинейно с постоянной скоростью.

Скорость нейтрона выражается через энергию:
\[
v(E)=\sqrt{\frac{2E}{m_n}},
\]
где $m_n$ — масса нейтрона.

Математически перенос описывается оператором:
\[
\boldsymbol{\Omega}\cdot\nabla\Psi
=
\Omega_x\frac{\partial\Psi}{\partial x}
+
\Omega_y\frac{\partial\Psi}{\partial y}
+
\Omega_z\frac{\partial\Psi}{\partial z}.
\]

\subsection{Захват нейтронов}

Захват — процесс, при котором нейтрон захватывается ядром
и исчезает как свободная частица.

\[
n + A \longrightarrow A' \longrightarrow A'' + n_1 \cdot \gamma, \qquad E' > E,
\]
\[
n + A \longrightarrow A' \longrightarrow A'' + n_2 \cdot e^{\pm}, \qquad E' > E,
\]
\[
n + A \longrightarrow A' \longrightarrow A'' + n_3 \cdot \alpha, \qquad E' > E.
\]

Вероятность поглощения:
\[
P = N_{\alpha}\sigma_a,
\]
где $\sigma_a(E)$ — микросечение поглощения, $N_{\alpha}$ — число атомов вещества в $\text{см}^2$. 

\[
\sigma = \frac{1}{N_{\alpha}}\left(1 - \frac{N_{after}}{N_{before}}\right) - \text{определение дифференциального сечения захвата}.
\]

В уравнении переноса поглощение учитывается членом:
\[
-\Sigma_a(E)\Psi(\mathbf r,\boldsymbol{\Omega},E,t),
\]
где знак «$-$» отражает уменьшение числа нейтронов.

\subsection{Рассеяние нейтронов}

Рассеяние — процесс столкновения нейтрона с ядром,
при котором нейтрон остаётся свободным,
но изменяет направление и, возможно, энергию.

Существуют:
\begin{itemize}
    \item упругое рассеяние $n + A \longrightarrow A' \longrightarrow n + A,\qquad E' = E$,
    \item неупругое рассеяние $n + A \longrightarrow A' \longrightarrow n + A,\qquad E' < E$.
\end{itemize}

Дифференциальное сечение рассеяния:
\[
\sigma_s(E\to E',\boldsymbol{\Omega}\to\boldsymbol{\Omega'}),
\]
характеризует вероятность перехода нейтрона
из состояния $(E,\boldsymbol{\Omega})$ в $(E',\boldsymbol{\Omega'})$.

Интегральный член рассеяния в уравнении переноса имеет вид:
\[
\int_{4\pi}\!\!\int_0^\infty
\Sigma_s(E\to E',\boldsymbol{\Omega}\to\boldsymbol{\Omega'})
\Psi(\mathbf r,\boldsymbol{\Omega},E,t)
\,dE\,d\boldsymbol{\Omega}.
\]

\subsection{Деление нейтронов}

Деление — процесс, при котором тяжёлое ядро,
захватив нейтрон, распадается с испусканием нескольких новых нейтронов.

\[
n + A \longrightarrow A' \longrightarrow B + C + k n, \qquad U_{92}^{235} (k \approx 2.5)
\]

Среднее число нейтронов на акт деления:
\[
\nu=\langle n\rangle.
\]

Источник нейтронов деления описывается выражением:
\[
Q_f(E')=
\frac{1}{k}
\int_0^\infty
\nu(E)\Sigma_f(E)\Phi(E)\chi(E')\,dE,
\]
где:
\begin{itemize}
  \item $\Sigma_f(E)$ — макросечение деления;
  \item $\Phi(E)=\int_{4\pi}\Psi\,d\boldsymbol{\Omega}$ — скалярный поток;
  \item $\chi(E)$ — энергетический спектр нейтронов деления;
  \item $k$ — эффективный коэффициент деления.
\end{itemize}

\subsection{Экспериментальные методы определения сечений}

\textbf{Метод ослабления пучка}

Интенсивность пучка нейтронов убывает по закону:
\[
I(x)=I_0 e^{-\Sigma_t x}.
\]

Из измерений интенсивности $I(x)$ определяется $\Sigma_t$, а затем $\sigma_t = \dfrac{\Sigma_t}{N}$.

\textbf{Метод активации}

Метод активации основан на измерении активности радиоактивных ядер,
образующихся в результате поглощения нейтронов.

При облучении образца нейтронным потоком $\Phi$
происходит реакция:
\[
X + n \rightarrow X^*,
\]
где $X^*$ — радиоактивное ядро.

Число радиоактивных ядер $N^*$ изменяется по закону:
\[
\frac{dN^*}{dt} = \Phi\,\sigma_a\,N - \lambda N^*,
\]
где:
\begin{itemize}
  \item $\sigma_a$ — микросечение поглощения;
  \item $N$ — число ядер исходного изотопа;
  \item $\lambda$ — постоянная распада.
\end{itemize}

В стационарном режиме:
\[
\Phi\,\sigma_a\,N = \lambda N^*.
\]

Активность образца определяется выражением:
\[
A = \lambda N^*.
\]

Отсюда сечение поглощения:
\[
\sigma_a = \frac{A}{\Phi\,N}.
\]

\textbf{Дифференциальные эксперименты}

Измеряется дифференциальное сечение:
\[
\frac{d\sigma}{d\Omega},
\]
что позволяет восстановить угловое распределение рассеяния.

\textbf{Эксперименты по делению}

Экспериментально определяются величины:
\[
\sigma_f(E),\quad \nu(E),\quad \chi(E).
\]
\end{document}

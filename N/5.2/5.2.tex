\documentclass[../../main.tex]{subfiles}

\begin{document}
\section{Присоединенные функции Лежандра. Метод сферических гармоник в задачах переноса нейтронов с рассеянием.}

\subsection{Полиномы Лежандра}

При наличии осевой симметрии угловая зависимость определяется направляющим
косинусом
\[
\mu=\cos\theta.
\]

Полиномы Лежандра $P_n(\mu)$ определяются как решения задачи Штурма–Лиувилля:
\[
\frac{d}{d\mu}\left[(1-\mu^2)\frac{dP_n}{d\mu}\right]
+
n(n+1)P_n(\mu)=0.
\]

Полином Лежандра степени 
$n \in \mathbb{Z}$ можно представить через формулу Родрига в виде

\[
P_n(\mu) = \frac{1}{2^n n!} \frac{d^n\left[(\mu^2 - 1)^n\right]}{d\mu^n}.
\]

Они также могут быть вычислены по рекуррентной формуле (при $n \ge 1$):
\[
P_{n+1}(\mu) = \frac{2n+1}{n+1} \mu P_n (\mu) - \frac{n}{n+1} P_{n-1}(\mu).
\]

Причём первые две функции имеют вид:
\[
P_0 (\mu) = 1, \qquad P_1(\mu) = \mu.
\]
\textbf{Свойства полиномов Лежандра:}
\begin{itemize}
  \item ортогональность:
  \[
  \int_{-1}^{1}P_n(\mu)P_{n'}(\mu)\,d\mu
  =
  \frac{2}{2n+1}\delta_{nn'};
  \]
  \item нормировка: $P_n(1)=1$;
  \item чётность: $P_n(-\mu)=(-1)^nP_n(\mu)$;
  \item полнота в пространстве $L^2[-1,1]$.
\end{itemize}

Для полиномов Лежандра справедливо соотношение ортогональности
на сфере радиуса $R$:
\[
\int_R\int_{4\pi }
P_n(\cos\theta)\,P_k(\cos\theta)\,d\Omega
=
\begin{cases}
\dfrac{4\pi R^2}{2n+1}, & n = k, \\
0, & n \neq k,
\end{cases}
\]
где элемент телесного угла равен
\[
d\Omega = \sin\theta\,d\theta\,d\varphi.
\]

\textbf{Теорема (о разложении по полиномам Лежандра).}
Любая функция $f(\mu)\in L^2[-1,1]$ допускает разложение
\[
f(\mu)=\sum_{n=0}^{\infty}a_nP_n(\mu),
\quad
a_n=\frac{2n+1}{2}\int_{-1}^{1}f(\mu)P_n(\mu)\,d\mu.
\]

\subsection{Присоединённые функции Лежандра}

Для общего трёхмерного случая при $\mu \in [-1, 1]$ вводятся присоединённые функции Лежандра:
\[
P_n^m(\mu)
=
(1-\mu^2)^{m/2}\frac{d^m}{d\mu^m}P_n(\mu),
\quad m=0,1,\dots,n,
\]
\[
P_n^m(\mu) = 0, \qquad m > n.
\]

Они являются решениями уравнения:
\[
\frac{d}{d\mu}\left[(1-\mu^2)\frac{dP_n^m}{d\mu}\right]
+
\left[n(n+1)-\frac{m^2}{1-\mu^2}\right]P_n^m(\mu)=0.
\]

\textbf{Свойства присоединённых функций Лежандра:}
\begin{itemize}
  \item ортогональность по $\mu$ при фиксированном $m$;
  \item $P_l^0(\mu)=P_l(\mu)$;
  \item при фиксированном $m$
  \[
  \int_{-1}^{1} P_n^m (\mu) P_{n'}^m (\mu) d \mu = \frac{2}{2n+1}\frac{(n+m)!}{(n-m)!} \delta_{nn'}.
  \]
\end{itemize}

Для них справедливо рекурретное соотношение:
\[
P_{n+1}^{m} (\mu) = \frac{2n+1}{n-m+1} \mu P_n^{m} (\mu) - \frac{n+m}{n-m+1} P_{n-1}^m (\mu).
\]

\subsection{Сферические гармоники}

Сферические гармоники определяются формулами
\[
Y_n^{(m)}(\theta,\varphi)
=
Y_n^{(m)}(\mu,\varphi)
=
\begin{cases}
P_n^{m}(\mu)\sin(m\varphi), & m>0, \\[6pt]
P_n^{|m|}(\mu)\cos(|m|\varphi), & m\le 0.
\end{cases}
\]

Область определения:
\[
-1 \le \mu \le 1,
\qquad
0 \le \varphi \le 2\pi.
\]

Сферические гармоники образуют полную ортогональную систему на сфере:
\[
\int_{0}^{2\pi}\!\!\int_{-1}^{1}
Y_{n_1}^{(m_1)}(\mu,\varphi)\,
Y_{n_2}^{(m_2)}(\mu,\varphi)
\,d\mu\,d\varphi
=
\begin{cases}
\dfrac{2\pi(2-\delta_{m0})(n+|m|)!}
{(2n+1)(n-|m|)!},
& n_1=n_2,\; m_1=m_2, \\[10pt]
0, & \text{иначе}.
\end{cases}
\]

Здесь символ Кронекера задаётся как
\[
\delta_{m0} =
\begin{cases}
1, & m=0,\\
0, & m\neq 0.
\end{cases}
\]

Элемент телесного угла может быть записан в эквивалентных формах:
\[
d\mathbf{\Omega}
=
\sin\theta\,d\theta\,d\varphi
=
d(\cos\theta)\,d\varphi
=
d\mu\,d\varphi.
\]

Любая достаточно гладкая функция направления
$f(\mathbf{\Omega})$ может быть разложена в ряд по сферическим гармоникам:
\[
f(\mathbf{\Omega})
=
\sum_{n=0}^{\infty}
\sum_{m=-n}^{n}
A_{nm}\,
Y_n^{(m)}(\mu,\varphi).
\]

Коэффициенты разложения определяются формулой
\[
A_{nm}
=
\frac{1}{N_{nm}}
\int_{0}^{2\pi}\!\!\int_{-1}^{1}
f(\Omega)\,
Y_n^{(m)}(\mu,\varphi)
\,d\mu\,d\varphi.
\]

Нормировочный множитель имеет вид
\[
N_{nm}
=
\frac{2\pi(2-\delta_{m0})}{2n+1}
\frac{(n+|m|)!}{(n-|m|)!}.
\]

Разложение функции по сферическим гармоникам может быть записано
в компактной нормированной форме:
\[
f(\mathbf{\Omega})
=
\sum_{n=0}^{\infty}
\sum_{m=-n}^{n}
\frac{2n+1}{2\pi(2-\delta_{m0})}
\frac{(n-|m|)!}{(n+|m|)!}
C_{nm}
Y_n^{(m)}(\mu,\varphi), где
\]

\[
C_{nm} = \int_{0}^{2 \pi}d\varphi \int_{-1}^{1} f(\mathbf{\Omega}) Y_n^{(m)} (\mu, \varphi) d \mu.
\]

Если функция $f(\Omega)$ не зависит от азимутального угла $\varphi$,
то разложение упрощается и содержит только полиномы Лежандра:
\[
f(\mu)
=
\sum_{n=0}^{\infty}
\frac{2n+1}{2}
c_n P_n(\mu),
\]
где коэффициенты находятся по формуле
\[
c_n
=
\int_{-1}^{1}
f(\mu)P_n(\mu)\,d\mu.
\]

\subsection{Нестационарное уравнение переноса}

Общее уравнение переноса имеет вид:
\[
\frac{1}{v}
\frac{\partial \Psi(\mathbf{r},t,\mathbf{\Omega},E)}{\partial t}
+
\mathbf{\Omega}\cdot\nabla
\Psi(\mathbf{r},t,\mathbf{\Omega},E)
+
\Sigma_t(\mathbf{r},E)\Psi(\mathbf{r},t,\mathbf{\Omega},E)
=
\]
\[
=
\frac{\Sigma_s(\mathbf{r},E)}{4\pi}
\int_{4\pi}
\omega(\mu_0)
\Psi(\mathbf{r},t,\mathbf{\Omega'},E)\,d\mathbf{\Omega'}
+
\frac{Q(\mathbf{r})}{4\pi}.
\]

Угловую плотность потока можно представить в виде ряда:
\[
\Psi(\mathbf{r},t,\mathbf{\Omega},E)
=
\sum_{n=0}^{\infty}
\sum_{m=-n}^{n}
\frac{2n+1}{2\pi(1+\delta_{m0})}
\frac{(n-|m|)!}{(n+|m|)!}
\Psi_{nm}(\mathbf{r},t,E)
Y_n^{(m)}(\mu,\varphi).
\]

Рассматривается одномерное стационарное уравнение:
\[
\mu \frac{\partial \Psi(z,\mu)}{\partial z}
+
\Sigma_t(z)\Psi(z,\mu)
=
\frac{\Sigma_s(z)}{2}
\int_{-1}^{1}
\omega(\mu_0)\Psi(z,\mu')\,d\mu'
+
\frac{Q(z)}{2}.
\]

Интегрирование по $\varphi$ выполнено, так как задача осесимметрична. Искомая функция представляется в виде полиномов Лежандра:
\[
\Psi(z,\mu)
=
\frac{1}{2}\Psi_0(z)
+
\frac{3}{2}\mu\Psi_1(z)
+
\frac{5}{2}\frac{3\mu^2-1}{2}\Psi_2(z)
+
\ldots + \frac{2n + 1}{2} P_n (\mu) \Psi_n (\mu).
\]

Индикатриса рассеяния $\omega(\mu_0)$ раскладывается по полиномам Лежандра:
\[
\omega(\mu_0)
=
\sum_{k=0}^{\infty}
\omega_k
\frac{2k+1}{2}
P_k(\mu_0) = \sum_{k=0}^{\infty} \omega_k \frac{2k+1}{2} P_k (\mu) P_k (\mu'),
\]
где $\mu_0 = \Omega\cdot\Omega'$ — косинус угла рассеяния.

Уравнение умножается на $P_k(\mu)$ и интегрируется по $\mu\in[-1,1]$.
Используется тождество:
\[
\mu P_k(\mu)
=
\frac{k+1}{2k+1}P_{k+1}(\mu)
+
\frac{k}{2k+1}P_{k-1}(\mu).
\]

\textbf{Теорема сложения для полиномов Лежандра.}
\[
P_n(\mathbf{\Omega}\cdot\mathbf{\Omega'}) = P_n (\mu) P_n(\mu') + 2\sum_{m=1}^n \frac{(n-m)!}{(n+m)!} P_n^{(m)} (\mu) P_n^{(m)} (\mu') \cos{m} (\varphi - \varphi')
\]

В результате получается система:
\[
k\frac{d\Psi_{k-1}}{dz}
+
(k+2)\frac{d\Psi_{k+1}}{dz}
+
(2k+1)\Sigma_k\Psi_k
=
(2k+1)Q_k,
\]
где
\[
\Sigma_k = \Sigma_t - \Sigma_s\omega_k, \qquad \omega_k - \text{\textit{k}-ый член разложения по полиномам Лежандра.}
\]

\subsection{Приближение $P_1$}

В приближении $P_1$:
\[
\Psi(z,\mu)
=
\frac{1}{2}\Psi_0(z)
+
\frac{3}{2}\mu\Psi_1(z).
\]

Система уравнений принимает вид:
\[
\begin{cases}
\dfrac{d\Psi_1}{dz} + \Sigma_0\Psi_0 = Q_0, \\[0.5cm]
\dfrac{d\Psi_0}{dz} + 3\Sigma_1\Psi_1 = 0.
\end{cases}
\]

Отсюда получается диффузионное уравнение:
\[
-\frac{d}{dz}
\left(
D(z)\frac{d\Psi_0}{dz}
\right)
+
\Sigma_0\Psi_0
=
Q_0,
\qquad
D(z)=\frac{1}{3\Sigma_1(z)}.
\]

\subsection{Приближение $P_2$}

В приближении $P_2$:
\[
\Psi(z,\mu)
=
\frac{1}{2}\Psi_0
+
\frac{3}{2}\mu\Psi_1
+
\frac{5}{2}\frac{3\mu^2-1}{2}\Psi_2.
\]

Получается система:
\[
\begin{cases}
\dfrac{d\Psi_1}{dz} + \Sigma_0\Psi_0 = Q_0, \\[0.5cm]
\dfrac{d\Psi_0}{dz} + 2\dfrac{d\Psi_2}{dz} + 3\Sigma_1\Psi_1 = 0, \\[0.5cm]
2\dfrac{d\Psi_1}{dz} + 5\Sigma_2\Psi_2 = 0.
\end{cases}
\]

При увеличении порядка $P_N$ возрастает точность учёта
анизотропии углового распределения, однако увеличивается
размерность системы уравнений.

\end{document}

\documentclass[../../main.tex]{subfiles}

\begin{document}
\section{ Постановка задачи на нахождение максимального собственного значения в задачах переноса нейтронов. Доминантное отношение. Критические параметры.}
\subsection{Уравнение переноса с делением}

Рассмотрим одномерное стационарное уравнение переноса нейтронов:
\[
\mu \frac{\partial \Psi(x,\mu)}{\partial x}
+
\Sigma_t(x)\Psi(x,\mu)
=
\frac{\Sigma_s(x)}{2}
\int_{-1}^{1}\Psi(x,\mu')\,d\mu'
+
\frac{1}{k}\,
\frac{\nu\Sigma_f(x)}{2}
\int_{-1}^{1}\Psi(x,\mu')\,d\mu'.
\]

Здесь:
\begin{itemize}
  \item $\Psi(x,\mu)$ — угловая плотность потока нейтронов;
  \item $\mu$ — направляющий косинус;
  \item $\Sigma_t=\Sigma_a+\Sigma_s$ — полное макросечение;
  \item $\Sigma_s$ — макросечение рассеяния;
  \item $\nu\Sigma_f$ — макросечение деления;
  \item $k$ — эффективный коэффициент размножения нейтронов.
\end{itemize}

Коэффициент $k$ характеризует баланс нейтронов в системе:
\[
\begin{aligned}
k = 1 &\quad \text{критическое состояние (стационарный режим)}, \\
k < 1 &\quad \text{подкритическая система (затухание нейтронов)}, \\
k > 1 &\quad \text{надкритическая система (рост числа нейтронов)}.
\end{aligned}
\]

Рассматриваются однородные граничные условия:
\[
\Psi(0,\mu)=0, \quad \mu>0,
\]
\[
\Psi(X,\mu)=\Psi(X,-\mu), \quad \mu<0.
\]

\subsection{Операторная форма задачи}

Введём оператор переноса без деления:
\[
\mathcal{L}\Psi
=
\mu \frac{\partial \Psi}{\partial x}
+
\Sigma_t(x)\Psi
-
\frac{\Sigma_s(x)}{2}
\int_{-1}^{1}\Psi(x,\mu')\,d\mu'.
\]

Оператор деления:
\[
\mathcal{P}\Psi
=
\frac{\nu\Sigma_f(x)}{2}
\int_{-1}^{1}\Psi(x,\mu')\,d\mu'.
\]

Тогда уравнение переноса можно записать в виде:
\[
\mathcal{L}\Psi
=
\frac{1}{k}\mathcal{P}\Psi.
\]

\subsection{Спектральная задача}

Применяя оператор $\mathcal{L}^{-1}$, получаем:
\[
\Psi
=
\frac{1}{k}
\mathcal{L}^{-1}\mathcal{P}\Psi.
\]

Вводя обозначение
\[
\mathcal{A} = \mathcal{L}^{-1}\mathcal{P},
\]
получаем спектральную задачу:
\[
k \Psi = \mathcal{A}\Psi.
\]

Таким образом, задача на нахождение критического состояния
сводится к задаче на собственные значения оператора $\mathcal{A}$.

Оператор $\mathcal{A}$ обладает следующими свойствами:
\begin{itemize}
  \item линейный;
  \item ограниченный;
  \item положительно определённый.
\end{itemize}

Следовательно, к нему применима теорема Крейна--Рутмана.

Собственные значения оператора $\mathcal{A}$ образуют последовательность:
\[
k_1 > |k_2| > |k_3| > \ldots
\]

При этом собственная функция,
соответствующая $k_1$, положительна:
\[
\Psi_1(x,\mu) > 0.
\]

\textbf{Замечание.}
Оператор $\mathcal{L}^{-1}$ соответствует решению
задачи переноса без деления и описывает
итерационный процесс по рассеянию.

\subsection{Доминантное отношение}

Определим доминантное отношение:
\[
r = \frac{|k_2|}{k_1}, \qquad 0 \le r < 1.
\]

Оно характеризует скорость сходимости итерационного процесса
в задачах переноса с делением.

\[
\begin{aligned}
\Sigma_t X \ll 1 &\quad \Rightarrow \quad r \ \text{мало (быстрая сходимость)}, \\
\Sigma_t X \gg 1 &\quad \Rightarrow \quad r \to 1 \ \text{(медленная сходимость)}.
\end{aligned}
\]

Таким образом, доминантное отношение связано
с эффективностью итераций по рассеянию и делению.

\[
\mathcal{L} \Psi (x, \mu) = \mu \frac{\partial \Psi}{\partial x} = \Sigma_t (x) \varphi - \frac{\Sigma_s}{2} \int_{-1}^{1} \Psi (x, \mu') d\mu', 
\]
где член
\[
\frac{\Sigma_s}{2} \int_{-1}^{1} \Psi (x, \mu') d\mu'
\]
отвечает за рассеяние.

\subsection{Критические параметры}

Критическим состоянием системы называется такое состояние,
при котором эффективный коэффициент размножения равен единице:
\[
k_1 = 1.
\]

При этом система находится в стационарном режиме:
число нейтронов в системе в среднем не изменяется во времени.

Параметры задачи, при которых выполняется условие $k_1=1$,
называются критическими параметрами.
К ним относятся:
\begin{itemize}
  \item геометрические размеры системы (длина, радиус, форма);
  \item макросечения среды $\Sigma_t$, $\Sigma_s$, $\nu\Sigma_f$;
  \item распределение материалов в пространстве;
  \item граничные условия.
\end{itemize}

Изменение любого из этих параметров приводит к изменению
главного собственного значения $k_1$ и, следовательно,
к переходу системы в подкритическое или надкритическое состояние:
\[
\begin{aligned}
k_1 < 1 &\quad \text{подкритическое состояние}, \\
k_1 = 1 &\quad \text{критическое состояние}, \\
k_1 > 1 &\quad \text{надкритическое состояние}.
\end{aligned}
\]

В задачах расчёта ядерных реакторов задача определения
критических параметров формулируется как задача подбора
таких характеристик системы, при которых выполняется условие
\[
\det(\mathcal{I} - \mathcal{A}) = 0,
\]
что эквивалентно существованию нетривиального решения
спектральной задачи
\[
k\Psi = \mathcal{A}\Psi
\quad \text{при} \quad k=1.
\]

Таким образом, задача определения критических параметров
сводится к задаче на нахождение максимального собственного
значения оператора переноса с делением.

\end{document}

\documentclass[../../main/main.tex]{subfiles}

\begin{document}
\section{Основные понятия теории разностных схем}
\subsection{Основные понятия теории разностных схем}


Рассматривается дифференциальная задача (Д.З.) с начальными (Н.У.) и граничными условиями (Г.У.)
\[
Lu(x, t) = f(x, t), \qquad u(x, t) \in U,
\]
где $L$ — дифференциальный оператор.

Ей ставится в соответствие разностная задача (Р.З.)
\[
\mathcal{L} y = \mathcal{F},
\]
где $\mathcal{L}$ — разностный оператор, $y \in \mathcal{U}$ — сеточная функция.

Вводится равномерная сетка:
\[
x_m = m h, \qquad t^n = n \tau,
\]
\[
u(x,t) \;\longrightarrow\; [u]_{h\tau} = \{ u(mh,n\tau) \}.
\]


\subsection{Норма и пространство сеточных функций}

В пространстве сеточных функций $\mathcal{U}$ вводится норма $\|\cdot\|_{\mathcal{U}}$, удовлетворяющая аксиомам:
\begin{enumerate}
  \item $\|y\|_{\mathcal{U}} \ge 0,\quad \|y\|_{\mathcal{U}} = 0 \iff y = 0$;
  \item $\|\alpha y\|_{\mathcal{U}} = |\alpha| \, \|y\|_{\mathcal{U}}$;
  \item $\|y + z\|_{\mathcal{U}} \le \|y\|_{\mathcal{U}} + \|z\|_{\mathcal{U}}$.
\end{enumerate}

\subsection{Сходимость разностной схемы}

\textbf{Определение.}
Решение разностной задачи сходится к решению дифференциальной задачи, если
\[
\|[u]_{h\tau} - y \|_{\mathcal{U}} \xrightarrow[h,\tau \to 0]{} 0.
\]

Если
\[
\|[u]_{h\tau} - y\|_\mathcal{U} \le C_1 h^p + C_2 \tau^q ,
\]
то порядок сходимости совпадает с порядком аппроксимации.


\subsection{Аппроксимация разностной схемы}

\textbf{Определение.}
Разностная задача (Р.З.) аппроксимирует дифференциальную задачу (Д.З.) на её решении, если
\[
r_{h\tau} = \mathcal{L} [u]_{h\tau} - \mathcal{F} \xrightarrow[h,\tau \to 0]{} 0,
\]
где $r_{h\tau}$ — невязка разностной задачи.

Если выполняется оценка
\[
\|r_{h\tau}\|_{\mathcal{U}} \le C_3 h^p + C_4 \tau^q,
\]
то говорят, что схема имеет порядок аппроксимации $p$ по $h$ и $q$ по $\tau$.

\subsection{Устойчивость разностной схемы}

\textbf{Определение.}
Разностная задача (Р.З.) называется устойчивой, если для любых возмущений правой части
\[
\mathcal{L} y^{(1)} = \mathcal{F} + \varepsilon^{(1)}, \qquad
\mathcal{L} y^{(2)} = \mathcal{F} + \varepsilon^{(2)},
\]
выполняется оценка
\[
\|y^{(1)} - y^{(2)}\|_\mathcal{U}
\le
C \left(
\|\varepsilon^{(1)}\|_U + \|\varepsilon^{(2)}\|_U
\right),
\]
где константа $C$ не зависит от $h$ и $\tau$.

\subsection{Корректность дифференциальной задачи}

\textbf{Определение (Адамар).}
Дифференциальная задача (Д.З.) называется корректно поставленной, если:
\begin{enumerate}
  \item решение существует;
  \item решение единственно;
  \item решение устойчиво по отношению ко входным данным.
\end{enumerate}

\subsection{Основная теорема вычислительной математики}

\textbf{Теорема (Лакса-Ребенького).}
Если разностная задача (Р.З.) аппроксимирует корректно поставленную дифференциальную задачу (Д.З.) на её решении и является устойчивой, то решение разностной задачи сходится к решению дифференциальной задачи. При этом порядок сходимости равен порядку аппроксимации.

\subsection{Замечание}

Для линейного оператора $\mathcal{L}$ условия устойчивости эквивалентны оценке
\[
\|y\|_\mathcal{U} \le C \|\mathcal{F}\|_\mathcal{U}.
\]

\end{document}

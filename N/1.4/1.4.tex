\documentclass[../../main.tex]{subfiles}

\begin{document}
\section{Первое дифференциальное приближение (ПДП) разностной схемы}
\subsection{Понятие первого дифференциального приближения (ПДП)}

Рассмотрим линейное уравнение переноса
\[
\frac{\partial u}{\partial t} + a \frac{\partial u}{\partial x} = 0,
\qquad a = \text{const},
\]
и разностную схему, аппроксимирующую данное уравнение,
\[
\mathcal{L} y = 0.
\]

\textbf{Определение.}
Первым дифференциальным приближением (ПДП) разностной схемы называется дифференциальное уравнение, получаемое из разностной схемы путём:
\begin{enumerate}
  \item подстановки точного гладкого решения $u(x,t)$ вместо сеточной функции $y_m^n$;
  \item разложения всех разностных выражений в ряд Тейлора;
  \item отбрасывания членов более высокого порядка малости.
\end{enumerate}

ПДП описывает \textbf{не исходное уравнение}, а то дифференциальное уравнение, которое \emph{фактически решает} разностная схема.

\subsection{ПДП для схемы «явный уголок»}

Рассмотрим схему «явный уголок» при $a>0$:
\[
\frac{y_m^{n+1} - y_m^n}{\tau}
+
a \frac{y_m^n - y_{m-1}^n}{h}
= 0.
\]

Подставляем точное решение $u(x,t)$:
\[
u(x_m,t^{n+1})
=
u + \tau u_t + \frac{\tau^2}{2}u_{tt} + O(\tau^3),
\]
\[
u(x_m-h,t^n)
=
u - h u_x + \frac{h^2}{2}u_{xx} + O(h^3).
\]

Подставляя в схему:
\[
\frac{u + \tau u_t + \frac{\tau^2}{2}u_{tt} - u}{\tau}
+
a \frac{u - (u - h u_x + \frac{h^2}{2}u_{xx})}{h}
= 0.
\]

Упрощая, получаем:
\[
u_t + \frac{\tau}{2}u_{tt}
+
a u_x
-
\frac{a h}{2}u_{xx}
+ O(\tau^2 + h^2)
= 0.
\]

Отбрасывая члены более высокого порядка, получим \textbf{гиперболическую форму ПДП}:
\[
\boxed{
u_t + a u_x = \frac{ah}{2} u_{xx} - \frac{\tau}{2} u_{tt}.}
\]

Используя дифференциальное продолжение:
\[
u_t = -a u_x,
\qquad
u_{tt} = a^2 u_{xx}.
\]

Подставляя, получим \textbf{параболическую форму ПДП}:
\[
\boxed{
u_t + a u_x
=
\left(
\frac{a h}{2}
-
\frac{a^2 \tau}{2}
\right) u_{xx}.}
\]

Получаем ПДП:
\[
u_t + a u_x
=
\mu u_{xx},
\qquad
\mu = \frac{a h}{2}(1-\sigma),
\qquad
\sigma = \frac{a\tau}{h},
\]
где $\sigma$ -- число Куранта, $\mu$ -- коэффициент схемной вязкости.

\subsection{Устойчивость как корректность задачи для ПДП.}

Рассмотрим задачу Коши для первого дифференциального приближения. Корректность данной задачи определяется знаком коэффициента $\mu$. При $\mu \ge 0$ уравнение является параболическим (или гиперболическим при $\mu=0$), и задача Коши корректно поставлена по Адамару, то есть решение существует, единственно и непрерывно зависит от начальных данных. В случае $\mu<0$ уравнение становится обратно-параболическим, что приводит к экспоненциальному росту малых возмущений и, следовательно, к некорректности задачи.

Таким образом, необходимым условием корректности ПДП является неравенство
\[
\mu \ge 0.
\]

Подставляя выражение для коэффициента $\mu$, получаем
\[
\frac{a h}{2}(1-\sigma) \ge 0.
\]

Поскольку при $a>0$ и $h>0$ выполняется неравенство
\[
\frac{a h}{2} > 0,
\]
знак коэффициента $\mu$ определяется только выражением $(1-\sigma)$. Отсюда следует условие
\[
1-\sigma \ge 0 \quad \Longleftrightarrow \quad \sigma \le 1,
\]
или
\[
\boxed{\frac{a\tau}{h} \le 1.}
\]

Следовательно, при выполнении условия Куранта $\sigma \le 1$ первое дифференциальное приближение схемы «явный уголок» является корректно поставленной задачей по Адамару.

\textbf{Утверждение.}
Если первое дифференциальное приближение
\begin{itemize}
  \item является корректно поставленной задачей по Адамару;
  \item имеет коэффициент диффузии $\mu \ge 0$,
\end{itemize}
то соответствующая разностная схема является устойчивой.

Из выполненных рассуждений следует, что при условии
\[
\frac{a\tau}{h} \le 1
\]
схема «явный уголок» устойчива.

\end{document}

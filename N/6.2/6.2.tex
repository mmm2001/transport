\documentclass[../../main.tex]{subfiles}

\begin{document}
\section{Многогрупповое приближение}

В задачах переноса нейтронов или фотонов функция распределения частиц
зависит от энергии $E$. При этом макросечения взаимодействия
$\Sigma_t(E)$, $\Sigma_s(E)$, $\Sigma_a(E)$ обладают сложной,
часто резонансной энергетической зависимостью.

Прямая дискретизация энергии с высокой точностью требует чрезвычайно
большого числа узлов. Например, в спектроскопии атмосферы или в
нейтронном переносе число энергетических линий может достигать
$10^6$--$10^7$, что делает прямой расчёт невозможным.

Для преодоления этой трудности вводится \emph{многогрупповое приближение},
в котором энергетическая переменная заменяется конечным числом
энергетических групп.

Рассмотрим одномерное стационарное уравнение переноса:
\[
\mu \frac{d \Psi(z,\mu,E)}{d z}
+ \Sigma_t(z,E)\Psi(z,\mu,E)
=
\frac{\Sigma_s(z,E)}{2}
\int_{-1}^{1}\omega(\mu_0)\Psi(z,\mu',E)\,d\mu'
+ \frac{Q(z,E)}{2}.
\]

Здесь:
\begin{itemize}
  \item $\Psi(z,\mu,E)$ — угловая плотность потока частиц (функция распределения);
  \item $\mu = \cos\theta$ — направляющий косинус угла между направлением движения частицы и осью $z$;
  \item $\mu'$ — направляющий косинус направления движения частицы до рассеяния;
  \item $\Sigma_t$, $\Sigma_s$ — полные и рассеивающие макросечения;
  \item $\omega(\mu_0)$ — индикатриса рассеяния;
  \item $Q(z,E)$ — источник.
\end{itemize}

Энергетическая ось разбивается на $P$ непересекающихся интервалов:
\[
E_1 > E_2 > \dots > E_{P+1},
\qquad
E \in [E_{p+1}, E_p],
\quad p=1,\dots,P.
\]

Для каждой группы вводится \emph{групповая плотность потока}:
\[
\Psi^p(z,\mu)
=
\int_{E_{p+1}}^{E_p}
\Psi(z,\mu,E)\,dE.
\]

Интегрируем исходное уравнение переноса по энергии на интервале
$[E_{p+1}, E_p]$:
\[
\int_{E_{p+1}}^{E_p}
\left(
\mu \frac{d \Psi}{d z}
+ \Sigma_t(z,E)\Psi
\right)dE
=
\int_{E_{p+1}}^{E_p}
\left(
\frac{\Sigma_s(z,E)}{2}
\int_{-1}^{1}\omega\Psi\,d\mu'
+ \frac{Q(z,E)}{2}
\right)dE.
\]

Производную по $z$ можно вынести за знак интеграла:
\[
\mu \frac{d \Psi^p(z,\mu)}{d z}
+
\int_{E_{p+1}}^{E_p}
\Sigma_t(z,E)\Psi(z,\mu,E)\,dE
=
\frac{1}{2}
\int_{-1}^{1}
\int_{E_{p+1}}^{E_p}
\Sigma_s(z,E)\omega\Psi\,dE\,d\mu'
+ \frac{Q^p(z)}{2}.
\]

Для замыкания системы вводятся \emph{групповые макросечения}:
\[
\Sigma_t^p(z)
=
\frac{
\int_{E_{p+1}}^{E_p}
\Sigma_t(z,E)\Psi(z,\mu,E)\,dE
}{
\int_{E_{p+1}}^{E_p}
\Psi(z,\mu,E)\,dE
},
\]
аналогично определяются $\Sigma_s^p(z)$.

На практике в численных методах используется аппроксимация
энергетического спектра внутри группы:
\[
\Psi(z,\mu,E)
\approx
B_{pl}(\theta,z)\Psi^p(z,\mu),
\]
где $B_{pl}(\theta,z)$ — известный спектр.

В результате получаем систему $P$ уравнений переноса:
\[
\mu \frac{d \Psi^p(z,\mu)}{d z}
+
\Sigma_t^p(z)\Psi^p(z,\mu)
=
\frac{\Sigma_s^p(z)}{2}
\int_{-1}^{1}
\omega(\mu_0)\Psi^p(z,\mu')\,d\mu'
+
\frac{Q^p(z)}{2},
\quad p=1,\dots,P.
\]

Таким образом, непрерывная энергетическая зависимость заменяется
дискретным набором энергетических групп.

Многогрупповое приближение позволяет:
\begin{itemize}
  \item учитывать сложную энергетическую структуру сечений;
  \item существенно снизить размерность задачи;
  \item применять стандартные численные методы для каждой группы.
\end{itemize}

Точность метода определяется числом энергетических групп и качеством
аппроксимации спектра внутри каждой группы.

\end{document}

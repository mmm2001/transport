\documentclass[../../main.tex]{subfiles}

\begin{document}
\section{Трудности, возникающие при численном решении уравнения переноса}

В общем случае уравнение переноса частиц имеет вид:
\[
\frac{1}{v(E)}\frac{\partial \Psi(\mathbf r,\boldsymbol{\Omega},E,t)}{\partial t}
+
\boldsymbol{\Omega}\cdot\nabla \Psi(\mathbf r,\boldsymbol{\Omega},E,t)
+
\Sigma_t(\mathbf r,E)\Psi
=
\]
\[
=
\int_{4\pi}\!\!\int_0^\infty
\Sigma_s(\mathbf r,E\!\to\!E',\boldsymbol{\Omega}\!\to\!\boldsymbol{\Omega'})
\Psi(\mathbf r,\boldsymbol{\Omega},E,t)
\,dE\,d\boldsymbol{\Omega}
+
Q(\mathbf r,\boldsymbol{\Omega},E,t).
\]

Функция распределения зависит от координат, направления,
энергии и времени:
\[
\Psi = \Psi(\mathbf r,\boldsymbol{\Omega},E,t).
\]

\subsection{Большая размерность пространства переменных}

В стационарной задаче:
\[
\Psi = \Psi(x,y,z,\mu,\varphi,E),
\qquad \mu=\cos\theta.
\]

Число неизвестных при дискретизации:
\[
N_{\text{неизв}} =
N_x N_y N_z \cdot N_\mu N_\varphi \cdot N_E.
\]

Даже при умеренных сетках:
\[
N_{\text{неизв}} \sim 10^7\text{--}10^{10},
\]
что делает невозможным прямое обращение оператора переноса.

\subsection{Плохая сходимость итераций по рассеянию}

Используется итерационная схема:
\[
\boldsymbol{\Omega}\cdot\nabla \Psi^{(n+1)}
+
\Sigma_t \Psi^{(n+1)}
=
\Sigma_s \Phi^{(n)} + Q,
\]
где
\[
\Phi(\mathbf r,E)
=
\int_{4\pi}\Psi(\mathbf r,\boldsymbol{\Omega},E)\,d\boldsymbol{\Omega}.
\]

Сходимость определяется спектральным радиусом:
\[
\rho \approx \frac{\Sigma_s}{\Sigma_t}.
\]

При
\[
\Sigma_s \to \Sigma_t
\quad \Rightarrow \quad
\rho \to 1,
\]
число итераций стремится к бесконечности:
\[
N_{\text{ит}} \sim \frac{1}{1-\rho}.
\]

\subsection{Анизотропия рассеяния}

Индикатриса рассеяния раскладывается по полиномам Лежандра:
\[
\omega(\boldsymbol{\Omega}\cdot\boldsymbol{\Omega}')
=
\sum_{l=0}^\infty
\frac{2l+1}{4\pi}\,\omega_l
P_l(\boldsymbol{\Omega}\cdot\boldsymbol{\Omega}').
\]

При сильно анизотропном рассеянии коэффициенты $\omega_l$
медленно убывают:
\[
\omega_l \not\to 0 \quad \text{при } l\to\infty.
\]

В методе сферических гармоник:
\[
\Psi(\mathbf r,\boldsymbol{\Omega})
=
\sum_{l=0}^N\sum_{m=-l}^l
\psi_{lm}(\mathbf r)Y_l^m(\boldsymbol{\Omega}),
\]
что требует большого $N$ и приводит к осцилляциям решения
и потере неотрицательности.

\subsection{Энергетическая зависимость и многорезонансное поглощение}

Макросечения зависят от энергии:
\[
\Sigma_t(E) = N\sigma_t(E).
\]

Резонансное микросечение имеет вид:
\[
\sigma(E)
\sim
\frac{\Gamma_n\Gamma_\gamma}
{(E-E_r)^2 + (\Gamma/2)^2}.
\]

Для корректного описания требуется:
\[
\Delta E \ll \Gamma,
\quad
N_E \gg 1.
\]

Используется многогрупповое приближение:
\[
\Psi_g(\mathbf r,\boldsymbol{\Omega})
=
\int_{E_{g-1}}^{E_g}\Psi\,dE,
\]
но при этом возникает ошибка усреднения:
\[
\Sigma_{t,g}
\neq
\frac{\int_{E_{g-1}}^{E_g}\Sigma_t(E)\Psi(E)\,dE}
{\int_{E_{g-1}}^{E_g}\Psi(E)\,dE}.
\]

\subsection{Связь уравнения переноса с другими физическими процессами}

\subsubsection{Связь с теплопереносом}

\[
\rho c_p \frac{\partial T}{\partial t}
=
\nabla\cdot(k\nabla T)
+
\int_0^\infty \kappa(E)\Sigma_f(E)\Phi(E)\,dE.
\]

Температура влияет на сечения:
\[
\Sigma_t(E,T) = \Sigma_t(E,T_0)f_D(T).
\]

\subsubsection{Связь с выгоранием топлива}

\[
\frac{dN_i}{dt}
=
-\sigma_{a,i}\Phi N_i
+
\sum_j \lambda_{ji}N_j.
\]

\[
\Sigma_t = \sum_i N_i\sigma_{t,i}.
\]

\subsubsection{Связь с нейтронной кинетикой}

\[
\frac{1}{v}\frac{\partial \Psi}{\partial t}
+
\boldsymbol{\Omega}\cdot\nabla\Psi
+
\Sigma_t\Psi
=
\int\Sigma_s\Psi\,d\Omega'
+
\sum_i \chi_i C_i.
\]

\[
\frac{dC_i}{dt}
=
\beta_i\int\nu\Sigma_f\Phi\,dE
-
\lambda_i C_i.
\]

\subsubsection{Связь с гидродинамикой}

Рассматривается плоскопараллельная (или лагранжева) система координат.
Радиация описывается уравнением переноса, связанного с уравнениями
гидродинамики через источниковые члены.

\[
\frac{d\rho}{dt} + \rho\,\nabla \cdot \bf{u} = 0,
\]
где:
\begin{itemize}
\item $\rho$ — плотность вещества,
\item $\bf{u}$ — скорость среды.
\end{itemize}

\[
\rho \frac{d\bf{u}}{dt}
+ \nabla \left( p_e + p_i + p_\omega \right)
=
\int_0^\infty
\frac{\chi_\nu}{c}
\bf{W}_\nu \, d\nu.
\]

Здесь:
\begin{itemize}
\item $p_e$, $p_i$ — электронное и ионное давления,
\item $p_\omega$ — давление излучения,
\item $\chi_\nu$ — коэффициент взаимодействия излучения с веществом,
\item $\bf{W}_\nu$ — поток излучательной энергии,
\item $c$ — скорость света.
\end{itemize}

Правая часть описывает \textbf{обмен импульсом между излучением и веществом}.

\textbf{Уравнение энергии электронов}

\[
\rho \frac{d\varepsilon_e}{dt}
+
\nabla \cdot \boldsymbol{W}_e +
\left(p_e + \delta \rho \omega
\right)
\nabla \cdot \boldsymbol{u}
=
\rho Q_{ei}
+
\rho Q_e
+
Q_r.
\]

Здесь:
\begin{itemize}
\item $\varepsilon_e$ — удельная энергия электронов,
\item $\bf{W}_e$ — поток электронной энергии,
\item $Q_{ei}$ — электрон-ионный энергообмен,
\item $Q_e$ — внешние источники,
\item $Q_r$ — обмен энергией с излучением.
\end{itemize}

\textbf{Уравнение энергии ионов}
\[
\rho \frac{d\varepsilon_i}{dt}
+
\nabla \cdot \vec{W}_i
+
\left(
p_i + (1-\delta)\rho\omega
\right)
\nabla \cdot \vec{u}
=
\rho Q_{ie}.
\]


\textbf{Модель электрон-ионного релаксационного обмена}
\[
Q_{ie}
=
\frac{T_e - T_i}{\tau_{ei}},
\]
где:
\begin{itemize}
\item $T_e$, $T_i$ — температуры электронов и ионов,
\item $\tau_{ei}$ — характерное время релаксации.
\end{itemize}

\textbf{Источник излучательной энергии}

Спектральная плотность источника:
\[
Q_\nu
=
- \frac{\partial}{\partial z}
\left[
z \, F_\nu(z,\rho,T_e)
\right].
\]

\textbf{Полный вклад излучения в энергетику}
\[
Q_r
=
\int_0^\infty
\left(
2\chi_\nu^a V_\nu^0
-
2\chi_\nu^a V_\nu^{0,\mathrm{eq}}
\right)
d\nu
=
-
\int_0^\infty
\left(
\frac{\partial V_\nu^0}{\partial t}
+
\nabla \cdot \vec{W}_\nu
\right)
d\nu.
\]

\textbf{Моменты интенсивности излучения}

Нулевой и первый моменты:
\[
V_\nu^0(\boldsymbol{r},t)
=
\int_{4\pi}
I(\boldsymbol{r},t,\boldsymbol{\Omega},\nu)\, d\boldsymbol{\Omega},
\]
\[
\boldsymbol{W}_\nu(\boldsymbol{r},t)
=
\int_{4\pi}
\boldsymbol{\Omega}\,
I(\boldsymbol{r},t,\boldsymbol{\Omega},\nu)\, d\boldsymbol{\Omega}.
\]

\textbf{Уравнение переноса излучения}
\[
\frac{1}{c}\frac{\partial I}{\partial t}
+
\boldsymbol{\Omega}\cdot\nabla I
+
\left(
\chi_\nu^a + \chi_\nu^s
\right) I
=
\frac{\chi_\nu^s}{4\pi}
\int_{4\pi}
\omega(\boldsymbol{\Omega}\cdot\boldsymbol{\Omega}')
I(\boldsymbol{r},\boldsymbol{\Omega}',t,\nu)\,
d\boldsymbol{\Omega}'
+
\chi_\nu^a B_\nu.
\]

Здесь:
\begin{itemize}
\item $I$ — интенсивность излучения,
\item $\chi_\nu^a$, $\chi_\nu^s$ — коэффициенты поглощения и рассеяния,
\item $\omega(\mu)$ — фазовая функция рассеяния,
\item $B_\nu$ — функция Планка.
\end{itemize}

\subsection{Требования к разностной схеме}

Разностная схема должна удовлетворять следующим условиям:

\textbf{Устойчивость}
\[
\|\Psi^n\| \le C\|\Psi^0\|.
\]

\textbf{Сходимость}
\[
\lim_{h\to 0}\|\Psi_h - \Psi\| = 0.
\]

\textbf{Неотрицательность}
\[
\Psi_h \ge 0.
\]

\textbf{Баланс частиц}
\[
\int_V \Sigma_a\Phi\,dV
=
\int_V Q\,dV
+
\int_{\partial V}\boldsymbol{J}\cdot d\mathbf S.
\]

\textbf{Корректная аппроксимация переноса}
\[
\boldsymbol{\Omega}\cdot\nabla\Psi
\approx
\frac{\Psi_i-\Psi_{i-\operatorname{sign}\mu}}{h}.
\]

\end{document}

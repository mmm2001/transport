\documentclass[../../main.tex]{subfiles}

\begin{document}
\section{Влияние пространственно-угловой дискретизации уравнения переноса на сходимость итерационного процесса по рассеянию для алмазной схемы численного решения стационарного уравнения переноса в плоскопараллельной геометрии. Кинетически-согласованные и несогласованные разностные схемы для кинетических уравнений низкой размерности (LO).}

Рассматривается стационарное уравнение переноса нейтронов:
\begin{equation*}
\mu \frac{\partial \Psi(x,\mu)}{\partial x}
+ \Sigma_t(x)\Psi(x,\mu)
=
\frac{\Sigma_s(x)}{2}\int_{-1}^{1}\Psi(x,\mu')\,d\mu'
+
\frac{Q(x)}{2},
\end{equation*}
где
\[
\Phi(x)=\int_{-1}^{1}\Psi(x,\mu)\,d\mu
\]
— скалярный поток,
$\mu=\cos\theta$ — угловая переменная,
$\Sigma_t=\Sigma_a+\Sigma_s$ — полное макросечение.

\

\subsection{Пространственно-угловая дискретизация}

Интегралы по $\mu$ аппроксимируются:
\[
\int_{-1}^{1} \Psi(x, \mu)\,d\mu
\approx
\sum_{n=1}^{N}\omega_n \Psi(x, \mu_n),
\]
где $\mu_n$ — корни многочлена Лежандра $P_N(\mu)$,
$\omega_n = \int_{-1}^{1}\ln{x}dx$ — веса квадратуры, $\ln(x) = \prod_{j=1}^{N}\dfrac{x-x_j}{x_n - x_j}$ — вспомогательные функции Лагранжа.



\subsection{Алмазная схема}

Интегрируя уравнение переноса по ячейке
$[x_{j-1/2},x_{j+1/2}]$,
получаем алмазную схему:
\begin{equation*}
\mu_n
\frac{\Psi(x_{j+1/2},\mu_n)-\Psi(x_{j-1/2},\mu_n)}{h_j}
+
\Sigma_{t}(x_j)\Psi(x_j, \mu_n)
=
\frac{\Sigma_{s}(x_j)}{2}\Phi(x_j)+\frac{Q(x_j)}{2}.
\end{equation*}

Скалярный поток:
\[
\Phi(x_j) = \sum_{n=1}^{N}\omega_n \Psi(x_j, \mu_n),
\qquad
\Psi(x_j, \mu_n)=\frac{1}{2}
\left(\Psi(x_{j+1/2},\mu_n)+\Psi(x_{j-1/2},\mu_n)\right).
\]

Алмазная схема является \textbf{кинетически несогласованной}.

\subsection{Метод итераций источниика}

\[\mu_n \dfrac{\Psi_{j+1/2, n}^{(\ell+1/2)} - \Psi_{j-1/2,n}^{(\ell+1/2)}}{h_j} + \Sigma_{t,j}\dfrac{1}{2}\left(\Psi_{j-1/2,n}^{(\ell+1/2)}+\Psi_{j+1/2,n}^{(\ell+1/2)}\right) = \dfrac{\Sigma_{s,j}}{2}\Phi_j^{(\ell)} + \dfrac{Q_j}{2}\]

Граничные условия:
\[
\Psi_{j+1/2,n}^{(\ell+1/2)} = \Psi_n^{in}, \qquad \mu_n >0,
\]
\[
\Psi_{J+1/2,n}^{(\ell+1/2)} = \Psi_{J+1/2,m}^{(\ell+1/2)}, \qquad \text{ где} \qquad \mu_n = -\mu_n < 0.
\]

\subsection{Метод бегущего счета (транспортный подход)}

Граничные условия:
\[
\Psi_{j+1/2,n}^{(\ell+1/2)} = \dfrac{(2\mu_n - \Sigma_{t,j}h_j)\Psi_{j-1/2,n}^{(\ell+1/2)}+h_j\left(\Sigma_{s,j}\Phi_j^{(\ell)}+Q_j\right)}{2\mu_n+\Sigma_{t,j}h_j}, \qquad \mu_n>0: j =\dfrac{1}{2}\rightarrow\dfrac{3}{2}\rightarrow\cdots\rightarrow J+\dfrac{1}{2}
\]

\[
\Psi_{j-1/2,n}^{(\ell+1/2)} = \dfrac{(2|\mu_n| - \Sigma_{t,j}h_j)\Psi_{j+1/2,n}^{(\ell+1/2)}+h_j\left(\Sigma_{s,j}\Phi_j^{(\ell)}+Q_j\right)}{2|\mu_n| + |\Sigma_{t,j}|h_j}, \qquad \mu_n = - \mu_m < 0:J+\dfrac{1}{2} \rightarrow \cdots \rightarrow \dfrac{1}{2}
\]

\[
\Phi_j^{(\ell+1)}=\Phi_j^{(\ell+1/2)} = \sum_{n=1}^N \omega_n \Psi_{j,n}^{(\ell+1/2)} = \dfrac{1}{2} \sum_{n=1}^N \omega_n \left(\Psi_{j-1/2,n}^{(\ell+1/2)}+\Psi_{j+1/2,n}^{(\ell+1/2)}\right)
\]
Итерации до выполнения условия $||\Phi^{(\ell+1)} - \Phi^{(\ell)}|| \le \varepsilon$ - желаемая точность.

\subsection{Фурье анализ сходимости}
\[
\Phi_j^{(\ell)} = \omega^{\ell} e^{i\Sigma_t\lambda x_j}, \qquad
\Psi_{j,n}^{(\ell+1/2)} = \omega^{\ell} a_n e^{i\Sigma_t\lambda x_j},
\]
\[\Psi_{j+1/2,n}^{(\ell+1/2)} = \omega^{\ell}b_ne^{i\Sigma_t\lambda x_{j+1/2}}, \qquad \Phi_j^{(\ell+1/2)} = \omega^{\ell} \alpha e^{i\Sigma_t\lambda x_j}.
\]

Подставляем в уравнение переноса
\[
\mu_n \omega^{\ell}b_n \dfrac{e^{i\Sigma_t \lambda x_{j+1/2}} - e^{i\Sigma_t\lambda x_{j-1/2}}}{h_j} + \Sigma_{t,j} \omega^{\ell} a_{n} e^{i\Sigma_t \lambda x_j} = \dfrac{\Sigma_{s,j}}{2} \omega^{\ell}\alpha e^{i\Sigma_t\lambda x_j}
\]
\begin{enumerate}
    \item $2i\dfrac{\mu_n}{h} b_n \sin{\dfrac{\Sigma_t h \lambda}{2}} + \Sigma_t a_n = \dfrac{\Sigma_s}{2}$
    \item Алмазность схемы: $a_n = b_n \cos{\dfrac{\Sigma_t h \lambda}{2}}$
    \item $\Phi = \int \Psi d \mu \Rightarrow \alpha = a_n \omega_n$
    \item $\omega = \alpha \Rightarrow (\ell + 1/2 \rightarrow \ell + 1 )$
\end{enumerate}

Решение системы дает:
\[
a_n = \dfrac{c}{2} \dfrac{1}{1 + i \mu_n \Lambda}, \qquad \Lambda = \dfrac{2}{\Sigma_t h} \tan{\dfrac{\Sigma_t h \lambda}{2}},
\]
\[
\omega_{SI} = \alpha(\Lambda) = \dfrac{c}{2} \sum_{n=1}^N
\dfrac{\omega_n}{1 + \mu_n^2 \Lambda^2}, \qquad 2 = \int_{-1}^{1} d\mu = \sum_n \omega_k,
\]
\[
\alpha = \max_{\Lambda}{\alpha(\Lambda)} = \alpha(0) = \dfrac{c}{2} \sum_{n=1}^N\omega_n = c.
\]

\subsection{Исследование DSA-ускорения для несогласованной схемы}


\textbf{HO-задача (алмазная схема).}

На половинном шаге итерации по рассеянию решается уравнение:
\[
\mu \frac{\partial \Psi^{(\ell+1/2)}(x,\mu)}{\partial x}
+ \Sigma_t \Psi^{(\ell+1/2)}(x,\mu)
=
\frac{\Sigma_s}{2}\,\Phi^{(\ell)}(x)
+ \frac{Q(x)}{2},
\]
где
\[
\Phi^{(\ell)}(x) = \int_{-1}^{1} \Psi^{(\ell)}(x,\mu)\,d\mu
\]
— скалярный поток на предыдущей итерации.

\textbf{LO-задача DSA (диффузионное ускорение).}

Для ускорения сходимости вводится диффузионная задача для поправки
\(
F^{(\ell+1)}(x)
\):
\[
- \frac{1}{3}\frac{\partial}{\partial x}
\left(
\frac{1}{\Sigma_t(x)}
\frac{\partial F^{(\ell+1)}(x)}{\partial x}
\right)
+
\Sigma_a(x) F^{(\ell+1)}(x)
=
\Sigma_s(x)\bigl(
\Phi^{(\ell+1/2)}(x) - \Phi^{(\ell)}(x)
\bigr).
\]

После решения LO-задачи выполняется коррекция скалярного потока:
\[
\phi^{(\ell+1)}(x) = \Phi^{(\ell+1/2)}(x) + F^{(\ell+1)}(x).
\]

Вводится равномерная сетка:
\[
x_{j\pm 1/2}, \qquad h_j = x_{j+1/2} - x_{j-1/2}.
\]

Интегрирование диффузионного уравнения по ячейке даёт:
\[
\int_{x_{j-1/2}}^{x_{j+1/2}} \Sigma_a F\,dx =
h_j \Sigma_{a,j} F_j + h_{j+1} \Sigma_{a,j+1} F_{j+1},
\]
получаем дискретное LO-уравнение:
\[
-\frac{1}{3}
\left[
\frac{1}{\Sigma_{t,j+1}}
\frac{F_{j+3/2}-F_{j+1/2}}{h_{j+1}}
-
\frac{1}{\Sigma_{t,j}}
\frac{F_{j+1/2}-F_{j-1/2}}{h_{j}}
\right]
+
\left[\dfrac{\Sigma_{a,j+1/2}h_{j+1} + \Sigma_{a,j}h_j}{h_{j+1}+ h_j}\right] h_{j+1/2} F_{j+1/2}
=
\]
\[
= \dfrac{1}{2} \left[\Sigma_{s,j+1}h_{j+1}\left(\Phi_{j+1}^{(\ell+1/2)} - \Phi_{j+1}^{(\ell)}\right) + \Sigma_{s,j}h_{j}\left(\Phi_{j}^{(\ell+1/2)} - \Phi_{j}^{(\ell)}\right)\right].
\]
\[
\Phi_j^{(\ell+1)} = \Phi_{j}^{(\ell+1/2)} + \dfrac{1}{2}\left(F_{j+1/2}^{(\ell+1)}+F_{j-1/2}^{(\ell+1)}\right).
\]
\textbf{Фурье-анализ в бесконечной среде.}

Предполагается равномерная сетка:
\[
h = \text{const}, \qquad Q(x)=0.
\]

Ищем решение в виде одной Фурье-гармоники:
\[
F_{j+1/2}^{(\ell+1)}
=
\omega^\ell
\beta
e^{i\Sigma_t\lambda x_{j+1/2}},
\qquad
\beta
=
\frac{c(1-\alpha)\cos\!\left(\frac{\lambda \Sigma_t h}{2}\right)}
{1-c+\frac{1}{3}
\left(
\frac{\sin(\lambda \Sigma_t h/2)}{\lambda \Sigma_t h/2}
\right)^2},
\]
где
\[
c = \frac{\Sigma_s}{\Sigma_t}
\]
— отношение рассеяния.

\[
\omega
= \alpha-
\frac{c(1-\alpha)\cos^2\!\left(\frac{\lambda \Sigma_t h}{2}\right)}
{1-c+\frac{1}{3}
\left(
\frac{\sin(\lambda \Sigma_t h/2)}{\lambda \Sigma_t h/2}
\right)^2}
\]

\textbf{Фактор сходимости.}

Итерационный множитель для DSA имеет вид:
\[
\omega(\lambda)
=
\alpha
-
\frac{c(1-\alpha)}
{1-c+\frac{1}{3}\lambda^2},
\]
где параметр
\[
\lambda
=
\frac{2}{\Sigma_t h}
\arctan\left(\frac{\Lambda \Sigma_t h}{2}\right)
\]
связан с пространственной дискретизацией.

Минимальное и максимальное значения:
\[
\omega_{\min}(\lambda)
=
\alpha
-
\frac{c(1-\alpha)}{1-c+\frac{1}{3}\Lambda^2},
\qquad
\omega_{\max}(\lambda)
=
\alpha
=
\omega_{\text{SI}}.
\]

\subsection{Кинетически согласованные разностные схемы для LO-части}

\textbf{Алмазная HO-схема.}

На полуитерации по рассеянию решение
\(
\Psi^{(\ell+1/2)}_{j\pm 1/2,n}
\)
считается известным.
Вводится ошибка:
\[
f^{(\ell+1/2)}_{j\pm 1/2,n}
=
\Psi_{j\pm 1/2,n} - \Psi^{(\ell+1/2)}_{j\pm 1/2,n}
.
\]

Вычитая точное уравнение из дискретного, получаем уравнение для ошибки:
\[
\mu_n
\frac{f_{j+1/2,n}
-
f_{j-1/2,n}}{h_j}
+
\Sigma_{t,j} f_{j,n}
-
\frac{\Sigma_{s,j}}{2}
\sum_{m=1}^{N} \omega_m f_{j,m}
=
\frac{\Sigma_{s,j}}{2}
\left(
\phi^{(\ell+1/2)}_j - \phi^{(\ell)}_j
\right),
\]
где
\[
f_{j,n}
=
\frac12
\left(
f^{(\ell+1/2)}_{j+1/2,n}
+
f^{(\ell+1/2)}_{j-1/2,n}
\right).
\]

При $\mu_n > 0$:
\[
f_{1/2,n} = 0.
\]

При $\mu_n = - \mu_m < 0$:
\[
f_{J+1/2,n} = f_{J+1/2, m}.
\]

Используется приближение $P_1$:
\[
f_{j\pm 1/2,n}
\approx
\frac12
\left(
F_{j\pm 1/2}
+
3 \mu_n G_{j\pm 1/2}
\right),
\]
где:
\[
F = \int_{-1}^{1} f\,d\mu,
\qquad
G = \int_{-1}^{1} \mu f\,d\mu.
\]

Домножим уравнение на $\omega_n$ и просуммируем по $n$:
\[
\frac{1}{h_j}
\left(
G^{(\ell+1)}_{j+1/2}
-
G^{(\ell+1)}_{j-1/2}
\right)
+
\Sigma_{a,j} F^{(\ell+1)}_j
=
\Sigma_{s,j}
\left(
\Phi^{(\ell+1/2)}_j - \Phi^{(\ell)}_j
\right).
\]

Домножим уравнение на $\mu_n \omega_n$ и просуммируем:
\[
\frac{1}{3h_j}
\left(
F^{(\ell+1)}_{j+1/2}
-
F^{(\ell+1)}_{j-1/2}
\right)
+
\Sigma_{t,j} G^{(\ell+1)}_j
=
0.
\]

Используются алмазные аппроксимации:
\[
F^{(\ell+1)}_j
=
\frac12
\left(
F^{(\ell+1)}_{j+1/2}
+
F^{(\ell+1)}_{j-1/2}
\right),
\]
\[
G^{(\ell+1)}_j
=
\frac12
\left(
G^{(\ell+1)}_{j+1/2}
+
G^{(\ell+1)}_{j-1/2}
\right).
\]

\textit{Условие нулевого падающего потока:}
\[
\sum_{\mu_n>0} \mu_n \left(F_{1/2}^{(\ell+1) }+3\mu_n G_{1/2}^{(\ell+1) }\right)\,\omega_n = 0.
\]

\textit{Зеркальное отражение:}
\[
G^{(\ell+1/2)}_{j+1/2} = 0.
\]

Из проекций HO-уравнения на базисы $P_0$ и $P_1$ получена система:
\begin{align}
G_{j+1/2} - G_{j-1/2}
+ \frac{\Sigma_{a,j} h_j}{2}
\left(
F_{j+1/2} + F_{j-1/2}
\right)
&=
\Sigma_{s,j} h_j
\left(
\Phi^{(\ell+1/2)}_j - \Phi^{(\ell)}_j
\right),
\tag{1}
\\[6pt]
G_{j+1/2} + G_{j-1/2}
+ \frac{2}{3}\Sigma_{t,j} h_j
\left(
F_{j+1/2} - F_{j-1/2}
\right)
&= 0.
\tag{2}
\end{align}

Складывая и вычитая уравнения (1)–(2), получаем:
\begin{equation}
G_{j+1/2}
=
-\frac{1}{3}\Sigma_{t,j} h_j
\left(
F_{j+1/2} - F_{j-1/2}
\right)
-
\frac{\Sigma_{a,j} h_j}{4}
\left(
F_{j+1/2} + F_{j-1/2}
\right)
+
\frac{\Sigma_{s,j} h_j}{2}
\left(
\Phi^{(\ell+1/2)}_j - \Phi^{(\ell)}_j
\right),
\tag{3}
\end{equation}

\[1 \le j \le J.\]

Аналогичное выражение для соседней ячейки
получается заменой $j \to j+1$:
\begin{align}
G_{j+1/2}
&=
-\frac{1}{3}\Sigma_{t,j+1} h_{j+1}
\left(
F_{j+3/2} - F_{j+1/2}
\right)
-
\frac{\Sigma_{a,j+1} h_{j+1}}{4}
\left(
F_{j+3/2} + F_{j+1/2}
\right)
\nonumber\\
&\quad
+
\frac{\Sigma_{s,j+1} h_{j+1}}{2}
\left(
\Phi^{(\ell+1/2)}_{j+1} - \Phi^{(\ell)}_{j+1}
\right).
\tag{4}
\end{align}

Приравнивая правые части (3) и (4),
получаем замкнутое разностное уравнение
для потоков $F_{j\pm 1/2}$:
\begin{align}
&-\frac{1}{3}\Sigma_{t,j+1} h_{j+1}
\left(
F_{j+3/2} - F_{j+1/2}
\right)
+
\frac{1}{3}\Sigma_{t,j} h_j
\left(
F_{j+1/2} - F_{j-1/2}
\right)
\nonumber\\
&\quad
-
\frac{\Sigma_{a,j+1} h_{j+1}}{4}
\left(
F_{j+3/2} + F_{j+1/2}
\right)
+
\frac{\Sigma_{a,j} h_j}{4}
\left(
F_{j+1/2} + F_{j-1/2}
\right)
\nonumber\\
&=
\frac{\Sigma_{s,j+1} h_{j+1}}{2}
\left(
\Phi^{(\ell+1/2)}_{j+1} - \Phi^{(\ell)}_{j+1}
\right)
+
\frac{\Sigma_{s,j} h_j}{2}
\left(
\Phi^{(\ell+1/2)}_{j} - \Phi^{(\ell)}_{j}
\right).
\tag{5}
\end{align}

После решения для $F$ скалярный поток обновляется по формуле:
\[
\phi^{(\ell+1)}_j
=
\phi^{(\ell+1/2)}_j
+
\frac12
\left(
F^{(\ell+1)}_{j-1/2}
+
F^{(\ell+1)}_{j+1/2}
\right).
\]

Полученная схема:
\begin{itemize}
\item не требует непрерывного диффузионного предела;
\item выводится строго из дискретного HO-уравнения;
\item сохраняет спектр ошибок переноса;
\item является \textbf{кинетически согласованной}.
\end{itemize}
\end{document}

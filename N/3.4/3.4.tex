\documentclass[../../main.tex]{subfiles}

\begin{document}
\section{Аддитивные методы ускорения итераций по рассеянию.\\ Диффузионное синтетическое ускорение (DSA) и Фурье-анализ его скорости сходимости на примере уравнения переноса \\в плоскопараллельной геометрии для непрерывного стационарного уравнения переноса.}

Рассмотрим одномерное стационарное уравнение переноса нейтронов
в плоскопараллельной геометрии:
\[
\mu \frac{\partial \Psi(x,\mu)}{\partial x}
+
\Sigma_t(x)\Psi(x,\mu)
=
\frac{\Sigma_s(x)}{2}
\int_{-1}^{1}\Psi(x,\mu')\,d\mu'
+
\frac{Q(x)}{2}.
\]

\textbf{Итерационная схема по рассеянию}

На $(\ell+1/2)$-й итерации решается уравнение:
\[
\mu \frac{\partial \Psi^{(\ell+1/2)}(x,\mu)}{\partial x}
+
\Sigma_t(x)\Psi^{(\ell+1/2)}(x,\mu)
=
\frac{\Sigma_s(x)}{2}
\int_{-1}^{1}\Psi^{(\ell)}(x,\mu')\,d\mu'
+
\frac{Q(x)}{2}.
\]

Граничные условия:
\[
\Psi^{(\ell+1/2)}(0,\mu)=\Psi^{\text{in}}(\mu), \quad \mu>0,
\]
\[
\Psi^{(\ell+1/2)}(X,\mu)
=
\Psi^{(\ell+1/2)}(X,-\mu), \quad \mu<0.
\]

Введём ошибки:
\[
f^{(\ell+1/2)}(x,\mu)
=
\Psi(x,\mu)-\Psi^{(\ell+1/2)}(x,\mu),
\]
\[
f^{(\ell)}(x,\mu)
=
\Psi(x,\mu)-\Psi^{(\ell)}(x,\mu).
\]

Для ошибки получаем уравнение:
\[
\mu \frac{\partial f^{(\ell+1/2)}(x,\mu)}{\partial x}
+
\Sigma_t(x)f^{(\ell+1/2)}(x,\mu)
=
\frac{\Sigma_s(x)}{2}
\int_{-1}^{1}
\left(
f^{(\ell+1/2)}(x,\mu')
-
f^{(\ell)}(x,\mu')
\right)d\mu'.
\]

\textbf{Аппроксимация ошибки по углам (P$_1$)}

Используется приближение:
\[
f(x,\mu)
=
\frac{1}{2}F(x)
+
\frac{3}{2}\mu G(x),
\]
где введены:
\[
F(x)=\int_{-1}^{1} f(x,\mu)\,d\mu,
\qquad
G(x)=\int_{-1}^{1} \mu f(x,\mu)\,d\mu.
\]

Интегрируя уравнение ошибки по $\mu$ на отрезке $[-1,1]$, получаем:
\[
\frac{\partial G(x)}{\partial x}
+
\Sigma_t(x)F(x)
=
\Sigma_s(x)F(x)
+
\Sigma_s(x)
\int_{-1}^{1}
\left(
\Psi^{(\ell+1/2)}-\Psi^{(\ell)}
\right)d\mu.
\]

С учётом $\Sigma_a=\Sigma_t-\Sigma_s$:
\[
\frac{\partial G(x)}{\partial x}
+
\Sigma_a(x)F(x)
=
\Sigma_s(x)
\left(
\Phi^{(\ell+1/2)}(x)
-
\Phi^{(\ell)}(x)
\right),
\]
где
\[
\Phi(x)=\int_{-1}^{1}\Psi(x,\mu)\,d\mu.
\]

Домножая уравнение ошибки на $\mu$ и интегрируя по $\mu$, получаем:
\[
\frac{1}{3}\frac{\partial F(x)}{\partial x}
+
\Sigma_t(x)G(x)
=
0.
\]

Отсюда:
\[
G(x)
=
-\frac{1}{3\Sigma_t(x)}\frac{\partial F(x)}{\partial x}.
\]

\textbf{Диффузионное уравнение для коррекции}

Подставляя $G(x)$ в первое уравнение, получаем:
\[
-\frac{\partial}{\partial x}
\left(
\frac{1}{3\Sigma_t(x)}
\frac{\partial F(x)}{\partial x}
\right)
+
\Sigma_a(x)F(x)
=
\Sigma_s(x)
\left(
\Phi^{(\ell+1/2)}(x)
-
\Phi^{(\ell)}(x)
\right).
\]

Это диффузионное уравнение для синтетической коррекции.


\textbf{Граничные условия}

Слева:
\[
\Psi(0,\mu)=\Psi^{\text{in}}(\mu),\quad \mu>0
\;\Rightarrow\;
f(0,\mu)=0.
\]

Отсюда:
\[
\int_0^1 f(0,\mu)\,d\mu
=
\frac{1}{4}F(0)+\frac{1}{2}G(0)=0.
\]

С учётом:
\[
G(0)
=
-\frac{1}{3\Sigma_t(0)}\frac{\partial F}{\partial x}(0),
\]
получаем граничное условие Маршака:
\[
F(0)
-
\frac{2}{3\Sigma_t(0)}
\frac{\partial F}{\partial x}(0)
=
0.
\]

Используем аппроксимацию ошибки:
\[
f(x,\mu) = \frac{1}{2}F(x) + \frac{3}{2}\mu G(x).
\]

Интегрируя по $\mu$ на отрезке $[-1,1]$, получаем:
\[
\int_{-1}^{1} f(X,\mu)\,d\mu
=
\frac{1}{2}F(X)\int_{-1}^{1} d\mu
+
\frac{3}{2}G(X)\int_{-1}^{1}\mu\,d\mu
=
F(X).
\]

С другой стороны:
\[
F(X) = \int_{-1}^{1}\Psi(X,\mu)\,d\mu.
\]

Рассмотрим интеграл с весом $\mu$:
\[
\int_{-1}^{1}\mu f(X,\mu)\,d\mu
=
\int_{-1}^{1}
\mu
\left(
\frac{1}{2}F(X)
+
\frac{3}{2}\mu G(X)
\right)d\mu
=
\frac{3}{2}G(X)\int_{-1}^{1}\mu^2\,d\mu
=
G(X).
\]

Из граничного условия симметрии:
\[
f(X,\mu) = f(X,-\mu), \quad \mu<0,
\]
следует:
\[
G(X)=0.
\]

С учётом ранее полученного соотношения:
\[
G(x)
=
-\frac{1}{3\Sigma_t(x)}\frac{\partial F(x)}{\partial x},
\]
получаем правое граничное условие:
\[
\frac{\partial F(X)}{\partial x} = 0.
\]

\textbf{Полная схема одной итерации DSA-метода}

Пусть задано начальное приближение:
\[
\Psi^{(0)}(x,\mu),
\qquad
\Phi^{(0)}(x)=\int_{-1}^{1}\Psi^{(0)}(x,\mu)\,d\mu.
\]

\textbf{Шаг 1. (HO-часть решения)}

Решаем уравнение переноса:
\[
\mu\frac{\partial \Psi^{(\ell+1/2)}(x,\mu)}{\partial x}
+
\Sigma_t(x)\Psi^{(\ell+1/2)}(x,\mu)
=
\frac{\Sigma_s(x)}{2}
\int_{-1}^{1}\Psi^{(\ell)}(x,\mu')\,d\mu'
+
\frac{Q(x)}{2},
\]
с граничными условиями:
\[
\Psi^{(\ell+1/2)}(0,\mu)=\Psi^{\text{in}}(\mu), \quad \mu>0,
\]
\[
\Psi^{(\ell+1/2)}(X,\mu)=\Psi^{(\ell+1/2)}(X,-\mu), \quad \mu<0.
\]

После решения вычисляется поток:
\[
\Phi^{(\ell+1/2)}(x)
=
\int_{-1}^{1}\Psi^{(\ell+1/2)}(x,\mu)\,d\mu.
\]

\textbf{Шаг 2. (LO-часть алгоритма, диффузионная поправка)}

Решается диффузионное уравнение для поправки:
\[
-\frac{\partial}{\partial x}
\left(
\frac{1}{3\Sigma_t(x)}
\frac{\partial F^{(\ell+1)}(x)}{\partial x}
\right)
+
\Sigma_a(x)F^{(\ell+1)}(x)
=
\Sigma_s(x)
\left(
\Phi^{(\ell+1/2)}(x)
-
\Phi^{(\ell)}(x)
\right),
\]
где
\[
\Sigma_a(x)=\Sigma_t(x)-\Sigma_s(x).
\]

Граничные условия:
\[
F^{(\ell+1)}(0)
-
\frac{2}{3\Sigma_t(0)}
\frac{\partial F^{(\ell+1)}(0)}{\partial x}
=
0,
\]
\[
\frac{\partial F^{(\ell+1)}(X)}{\partial x}
=
0.
\]

\textbf{Обновление решения}

Итерация завершается обновлением потока:
\[
\Phi^{(\ell+1)}(x)
=
\Phi^{(\ell+1/2)}(x)
+
F^{(\ell+1)}(x).
\]

\subsection{Исследование скорости сходимости DSA-алгоритма}

Рассмотрим одну пространственную фурье-гармонику:
\[
e^{i\Sigma_t\lambda x}.
\]

Будем полагать:
\[
\Phi^{(\ell)}(x)
=
\omega^{\ell}(\lambda)e^{i\Sigma_t\lambda x},
\]
\[
\Psi^{(\ell+1/2)}(x,\mu)
=
\omega^{\ell}(\lambda)\alpha(\lambda,\mu)e^{i\Sigma_t\lambda x},
\]
\[
\Phi^{(\ell+1/2)}(x)
=
\omega^{\ell}(\lambda)\beta(\lambda)e^{i\Sigma_t\lambda x},
\]
\[
F^{(\ell+1)} = \omega^{\ell}(\lambda) \gamma (\lambda)e^{i\Sigma_t\lambda x}.
\]

Подставляем в уравнение DSA-алгоритма:
\[
(i\lambda\mu + 1)\alpha(\lambda,\mu) = \frac{c}{2} \longleftrightarrow \text{HO - часть,}
\]
\[
\beta(\lambda) = \int_{-1}^{1}\alpha(\lambda, \mu)d\mu \longleftrightarrow \Phi^{(\ell+1/2)}(x) = \int_{-1}^{1}\Psi^{(\ell+1/2)}(x,\mu)d\mu,
\]
\[
\left(\frac{\lambda^2}{3}+1-c\right)\gamma(\lambda) = c|\beta(\lambda)-1|\longleftrightarrow \text{LO - диффузионное уравнение},
\]
\[
\omega(\lambda) = \beta(\lambda) + \gamma(\lambda).
\]

Решение системы:
\[
\alpha(\lambda, \mu) = \frac{c}{2} \frac{1}{1 + i \lambda\mu}, \qquad \beta (\lambda) = \frac{c}{\lambda}\arctan{\lambda}, 
\]
\[
\gamma(\lambda) = \frac{3c}{\lambda^2 + 3(1-c)}\left(\frac{c}{\lambda}\arctan{\lambda} - 1\right), \qquad \omega(\lambda) = \frac{3c}{\lambda^2 + 3(1-c)}\left[\left(\frac{\lambda^2}{3} + 1\right)\arctan{\lambda} - 1\right].
\]

\[
\sigma_{DSA} = \max_{\lambda}{|\omega(\lambda)|} \text{ - показатель скорости сходимости.}
\]

При отношении рассеяния $c = \dfrac{\Sigma_s}{\Sigma_t} = 1 \longrightarrow \sigma_{DSA}\approx 0.2247 c$.

Оценка необходимости применения DSA:
\[\dfrac{||\Psi^{(\ell)}-\Psi||}{||\Psi^{(0)} - \Psi||} \le 10^{-m} \Longrightarrow 0.1 \le c \le 1\].
\end{document}
